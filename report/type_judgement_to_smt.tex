\documentclass[12pt,a4paper,titlepage]{article}
\usepackage{mathtools}
\usepackage{stmaryrd}
\usepackage{amssymb}
\usepackage[margin=1in]{geometry}
\newcommand{\lnc}[1]{Logic(\langle \rangle \vdash #1)}
\newcommand{\triple}[3]{\langle \texttt{#1} , \texttt{#2} , \texttt{#3} \rangle}

\title{Type Checking The Refinement Typed Calculus With An SMT Solver}
\author{Charlie Stanton}

\begin{document}
    \maketitle

    \section{Starting Calculus}
    The language I defined earlier with $\gamma \coloneqq \{Bool\}$.

    \section{Translation to SMT}
    First I lambda lift $\lambda$-abstractions anywhere in the typing judgement.
    This can be done entirely before the translation to SMT happens using a relation $\Rightarrow_{LL}$ which I define.

    $\Gamma, x: \phi[\lambda y: \phi' . u], \Gamma' \vdash t: \psi \Rightarrow_{LL}
    \Gamma, f: \Pi_{y: \phi'} (z: \tau) z =_\tau u, x: \phi[f], \Gamma' \vdash t: \psi$

    Where $\tau = SimpleTypeOf(u)$.

    $\Gamma \vdash t[\lambda x: \phi . u]: \psi \Rightarrow_{LL}
    \Gamma, f: \Pi_{x: \phi} (y: \tau) y =_\tau u \vdash t[f]: \psi$

    Where $\tau = SimpleTypeOf(u)$.

    $\Gamma \vdash t: \psi[\lambda x: \phi . u] \Rightarrow_{LL}
    \Gamma, f: \Pi_{x: \phi} (y: \tau) y =_\tau u \vdash t: \psi[f]$

    Where $\tau = SimpleTypeOf(u)$.

    Next I define $Logic$ recursively.
    I'm using the SMT-LIB2 front-end syntax for my SMT program, but the implementation will be working with
    a library to interface with Z3 instead of generating a program as a string.

    $Logic(x: \phi, \Gamma \vdash t: \psi) \coloneqq$
    \begin{verbatim}
        (declare-const x T)
        (assert P)
        L
    \end{verbatim}
    Where $\triple{x}{T}{P} = Simplify(\phi)$ and $L = Logic(\Gamma \vdash t: \psi)$.

    $\lnc{t: \phi} \coloneqq$
    \begin{verbatim}
        (assert (not Q))
        (check-sat)
        (get-model)
    \end{verbatim}
    Where $Q = P[t/x]$ and $\triple{x}{\_}{P} = Simplify(\phi)$.

    I now define $Simplify$ which was used in the definition of $Logic$.
    Note that in the implementation, the simplified types will be in a recursive data structure and would
    be given unique identifiers later on.

    $Simplify(\textbf{1}) \coloneqq \triple{x}{one}{true}$

    $Simplify(T) \coloneqq \triple{x}{T}{true}$ where $T \in \gamma$

    $Simplify(\Sigma_{x:\phi} \psi) \coloneqq
    \triple{z}{TxU}{(and P[(first z) / x] Q[(first z) / x][(second z) / y])}$
    where $\triple{x}{T}{P} = Simplify(\phi)$ and $\triple{y}{U}{Q} = Simplify(\psi)$.

    $Simplify(\Pi_{x: \phi} \psi) \coloneqq
    \triple{f}{T->U}{(forall ((x T)) (=> P Q[(apply\_T->U f x) / y]))}$
    where $\triple{x}{T}{P} = Simplify(\phi)$ and $\triple{y}{U}{Q} = Simplify$.

    $Simplify((x: \phi)P) \coloneqq
    \triple{x}{T}{(and P' Q)}$
    where $\triple{x}{T}{Q} = Simplify(\phi)$ and $P' = Proposition(P)$.

    I now define $Proposition$.

    $Proposition(false) \coloneqq \texttt{false}$

    $Proposition(P \supset Q) \coloneqq \texttt{(=> P Q)}$
    where $\texttt{P} = Proposition(P)$ and $\texttt{Q} = Proposition(Q)$

    $Proposition(\forall x: \phi . P) \coloneqq \texttt{(forall ((x T)) (=> Q P))}$
    where $\triple{x}{T}{Q} = Simplify(\phi)$ and $\texttt{P} = Proposition(P)$

    $Proposition(F(t_1, ..., t_n)) \coloneqq \texttt{(F t1 ... tn)}$
    where $\texttt{ti} = Term(t_i)$

    $Proposition(t =_\textbf{1} t') \coloneqq P$
    where

    $\texttt{P} = \begin{cases}
        \texttt{true} & \text{if } SimpleTypeOf(t)=\textbf{1} \land SimpleTypeOf(t)=\textbf{1}\\
        \texttt{false} & \text{otherwise}
    \end{cases}$

    $Proposition(t =_T t') \coloneqq P$
    where $T \in \gamma$, $\texttt{t} = Term(t)$, $\texttt{t'} = Term(t')$ and

    $\texttt{P} = \begin{cases}
        \texttt{(= t t')} & \text{if } SimpleTypeOf(t)=T \land SimpleTypeOf(t)=T\\
        \texttt{false} & \text{otherwise}
    \end{cases}$

    $Proposition(t =_{\Sigma_{x: \phi} \psi} t') \coloneqq
    \texttt{(and P Q)}$ where
    $\texttt{P} = Proposition(\pi_1 (t) =_\phi \pi_2 (t'))$ and
    $\texttt{Q} = Proposition(\pi_2 (t) =_{\psi[t/x]} \pi_2 (t'))$

    $Proposition(t =_{\Pi_{x: \phi} \psi} t') \coloneqq
    \texttt{(forall ((x T)) (=> P Q))}$
    where $\triple{x}{T}{P} = Simplify(\phi)$ and $\texttt{Q} = Proposition(tx =_\psi t'x)$

    $Proposition(t =_{(x: \phi)P} t') \coloneqq
    \texttt{(and (and P[t/x] P[t'/x]) Q)}$
    where $\texttt{P} = Proposition(P)$ and
    $\texttt{Q} = Proposition(t =_\phi t')$

    $Proposition(\phi \sqsubseteq \psi) \coloneqq$

    $\begin{cases}
        \texttt{(forall ((x T)) (=> Q P))} & \text{if } T=U\\
        \texttt{false} & \text{otherwise}
    \end{cases}$

    where $\triple{x}{T}{P} = Simplify(\phi)$ and $\triple{x}{U}{Q} = Simplify(\psi)$

    I now define $Term$.

    $Term(x) \coloneqq \texttt{x}$

    $Term(\ast) \coloneqq \texttt{ast}$

    $Term(k(t_1, ..., t_n)) \coloneqq \texttt{(k t1 ... tn)}$ where $\texttt{ti} = Term(t_i)$

    $Term(\langle t, t' \rangle) \coloneqq \texttt{(pair t t')}$
    where $\texttt{t} = Term(t)$ and $\texttt{t'} = Term(t')$

    $Term(\pi_1 (t)) \coloneqq \texttt{(first t)}$ where $\texttt{t} = Term(t)$

    $Term(\pi_2 (t)) \coloneqq \texttt{(second t)}$ where $\texttt{t} = Term(t)$

    $Term(t\ t') \coloneqq \texttt{(apply\_T t t')}$
    where $\texttt{t} = Term(t)$, $\texttt{t'} = Term(t')$ and $\texttt{T} = SimpleTypeOf(t)$
\end{document}
