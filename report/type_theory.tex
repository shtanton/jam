\section{Type Checker Theory}

\subsection{Introducing The Calculus}

The type checker is easily the most complex aspect of the compiler, as the type system of \textsc{Refine} is
so much more expressive than most languages.
$\lambda^{\times \rightarrow \sqsubseteq}$, which \textsc{Refine} is based on,
is far more complex than the $\lambda^{\times \rightarrow}$
calculus even though it only introduces refinement types.

\subsection{Interesting Rules}

$\lambda^{\times \rightarrow \sqsubseteq}$ has 62 typing rules, which don't include the extra rules
that would need to be added for ground types like $bool$ and $u8$ to allow them to be useful in
refinement types, so I only pick out some interesting rules for discussion in Figure \ref{fig:interesting_type_rules}.

\begin{figure}
    \centering
    \begin{prooftree}
        \hypo{\Gamma \vdash t =_\phi t'}
        \infer1[(EQ)]{\Gamma \vdash t: \phi}
    \end{prooftree} \quad
    \begin{prooftree}
        \hypo{\Gamma \vdash t: \phi}
        \hypo{\Gamma \vdash P[t/x]}
        \infer2[(REFTYPE-INTRO)]{\Gamma \vdash t: (x: \phi)P}
    \end{prooftree}
    \\[5mm]
    \begin{prooftree}
        \hypo{\Gamma \vdash \phi' \sqsubseteq \phi}
        \hypo{\Gamma, x: \phi \vdash t =_\psi t'}
        \infer2[(ABS-$\xi$-EQ)]{\Gamma \vdash \lambda x: \phi . t =_{\Pi_{x:\phi} \psi} \lambda x: \phi' . t'}
    \end{prooftree} \quad
    \begin{prooftree}
        \hypo{\Gamma \vdash t: \phi}
        \infer1[(REFL-EQ)]{\Gamma \vdash t =_\phi t}
    \end{prooftree}
    \\[5mm]
    \begin{prooftree}
        \hypo{\Gamma \vdash t: (x: \phi)P}
        \infer1[(REFTYPE-ELIM)]{\Gamma \vdash P[t/x]}
    \end{prooftree} \quad
    \begin{prooftree}
        \hypo{\Gamma \vdash t: \Sigma_{x: \phi} \psi}
        \infer1[(PROJ2)]{\Gamma \vdash \pi_2(t): \psi[\pi_1(t)/x]}
    \end{prooftree}
    \caption{Interesting Type Rules}
    \label{fig:interesting_type_rules}
\end{figure}

(EQ) is useful when combined with another rule like (ABS-$\xi$-EQ).
Denney uses the example of proving $\lambda n: nat.n : even \rightarrow even$, which can't be done
without these rules.
We need to involve equality and logic for a proof like this as it is not the case that any function
with type $nat \rightarrow nat$ is also of type $even \rightarrow even$.
(REFTYPE-INTRO) and (REFTYPE-ELIM) show how refinement types function, and the logical assumption
that we're allowed to make given that a refinement type holds.
These are vital for grounding the abstract calculus in logic, and allows the translation to SMT
which I discuss later to happen.
I included (REFL-EQ) as it shows a problem with a syntax directed approach to type checking this
calculus.
The calculus has the strange property that any typing can be proved using an infinitely large
proof tree (which is of course invalid) that alternates between (REFL-EQ) and (EQ).
So we can't start at the base of a tree and grow upwards as we can always easily enter an infinite
loop.
Finally, (PROJ2) shows one of the dependent typing properties of the calculus, where the \textit{type}
of $\pi_2(t)$ depends on the \textit{value} of $\pi_1(t)$.

\subsection{Other Extensions}

As well as the typing rules, I have defined various sets that Denney left abstract in his calculus.
These are the ground types (Table \ref{tab:ground_types}), constants (Table \ref{tab:constants}),
predicates (Table \ref{tab:predicates}) and axioms, which describe the semantics of the other three
sets, but which I won't list exhaustively, instead just looking at one interesting example in
Figure \ref{fig:u8rec_axiom}.

\begin{table}
    \centering
    \begin{tabular}{|l|l|}
        \cline{1-2}
        $bool$ & Booleans\\\cline{1-2}
        $u8$ & Unsigned 8 bit integers\\\cline{1-2}
    \end{tabular}
    \caption{Ground Types}
    \label{tab:ground_types}
\end{table}

\begin{table}
    \centering
    \begin{tabular}{|l|l|p{0.508\textwidth}|}
        \cline{1-3}
        $true$, $false$ & $bool$ & Literals for the booleans.\\\cline{1-3}
        $\land$, $\lor$, $\Rightarrow$, $\Leftrightarrow$ & $bool, bool \rightarrow bool$ & Standard binary boolean operators.\\\cline{1-3}
        $\lnot$ & $bool \rightarrow bool$ & Unary not.\\\cline{1-3}
        $\textgreater$, $\textless$, $\leq$, $\geq$ & $u8, u8$ & Integer inequalities.\\\cline{1-3}
        $ite$ & $bool, \tau, \tau \rightarrow \tau$ & If-Then-Else.\\\cline{1-3}
        $0$, ..., $255$ & $u8$ & Literals for the 8 bit unsigned integers.\\\cline{1-3}
        $succ$, $pred$ & $u8 \rightarrow u8$ & The successor and predecessor of a $u8$, which will over/underflow.\\\cline{1-3}
        $+$, $-$ & $u8, u8 \rightarrow u8$ & Addition and subtraction for $u8$, which will over/underflow.\\\cline{1-3}
        $u8rec$ & $\tau, (u8 \rightarrow \tau \rightarrow \tau), u8 \rightarrow \tau$ &
        Recurse over the $u8$s less than a given $u8$ in ascending order.\newline
        $u8rec\ t\ t'\ 0 \coloneqq t$ \newline
        $u8rec\ t\ t'\ succ(n) \coloneqq t'\ n\ (u8rec\ t\ t'\ n)$\\\cline{1-3}
    \end{tabular}
    \caption{Constants}
    \label{tab:constants}
\end{table}

\begin{table}
    \centering
    \begin{tabular}{|l|l|l|}
        \cline{1-3}
        $bool$ & $bool$ & Interpret a boolean expression as a proposition.\\\cline{1-3}
        $\textgreater$, $\textless$, $\leq$, $\geq$ & $u8, u8$ & Integer inequalities.\\\cline{1-3}
    \end{tabular}
    \caption{Predicates}
    \label{tab:predicates}
\end{table}

\begin{figure}
    \centering
    \begin{prooftree}
        \hypo{\Gamma \vdash n: u8}
        \hypo{\Gamma \vdash t: \phi[0]}
        \hypo{\Gamma \vdash t': \Pi_{n:nat} \Pi_{y:\phi[n]} \phi[succ(n)]}
        \infer3[(U8REC)]{\Gamma \vdash u8rec\ t\ t'\ n: \phi[n]}
    \end{prooftree}
    \caption{u8rec Axiom Schema}
    \label{fig:u8rec_axiom}
\end{figure}

\subsection{u8 vs nat}

Denney's examples use $nat$ as the basic numeral type in his calculus, while I chose $u8$.
There are two main advantages, and one main disadvantage to $u8$.

\subsubsection{u8 Is Finite}

The calculus I chose needed to be able to be checked in an SMT solver (see later for why), which
constrained a lot of my decisions, as SMT solvers are still quite limited in what they can prove.
While testing, I found that frequently, Z3 was unable to reason well enough to avoid a brute force
approach.
This approach would work for $u8$, but for $nat$ the type checker would have to return unknown.
Even with $u8$, if enough of these brute force methods are needed at the same time, then the problem could easily
grow to beyond what reasonable hardware can possibly compute, however, by using $u8$ I at least
have the guarantee that any satisfiability will be decidable, which I can't have with $nat$.
It is also possible that a problem involving $nat$ could have a counterexample which is enormous
enough that the hardware the compiler is being run on can't compute it.

\subsubsection{u8 Is Easy For Hardware}

For code generation later on, implementing $u8$ took next to no time, and I can have a very high
degree of certainty that it works, because the type is simple enough to be available even in
assembly.
Implementing $nat$ would have required memory allocation on the heap, and garbage collection,
with more memory being allocated if the number outgrows the space allocated, with custom
implementations of basic integer operators that can work on them.
This would have given a lot of scope for error, and bugs in the compiler make the formal verification
worthless, as it no longer can be relied upon to have valid proofs.

\subsubsection{u8 Is Limited}

$u8$ can't represent a very large range of values, I only chose it for simplicity, but if \textsc{Refine}
were being written for use in industry, signed 32 or 64 bit integers would have made more sense.

\subsection{Implementing The Typing Rules}

Type checking for a standard language can be done using the typing rules of the calculus.
This is because the bottom judgement in a typing rule is unique to that typing rule, and if a judgement
fits the bottom of a rule, the requirements for the top rule can be generated from it.

For example, suppose we wish to show the typing $\langle \rangle \vdash \lambda x: u8 . x: u8 \rightarrow u8$
in the simply typed lambda calculus.
We try to construct a proof tree from the bottom up.
The bottom rule must be (ABS), which must require $x:u8 \vdash x : u8$.
Again only one rule can be used which is (VAR) to complete the tree.
This way we can easily construct a proof tree if it exists, or prove it doesn't exist for any judgement.

Now we try the same exercise using $\lambda^{\times \rightarrow \sqsubseteq}$,
supposing we want to prove that $\langle \rangle \vdash \lambda x: u8 . x : u8$ cannot have a proof tree.
The rules that could be at the bottom of our tree are (EQ) or (WEAKEN).
Both of these have a part of the top of the rule that isn't mentioned in the bottom.
To prove we can't use (EQ) we have to show that there exists no $t'$ such that
$\langle \rangle \vdash \lambda x: u8 . x =_{u8} t'$, which is extremely difficult.
To prove we can't use (WEAKEN) we have to show that there exists no $\phi'$ such that
$\langle \rangle \vdash \lambda x: u8 . x : \phi'$ and $\langle \rangle \vdash \phi \sqsubseteq \phi'$
which is very difficult also.

In fact, because of the first order logic embedded into
$\lambda^{\times \rightarrow \sqsubseteq}$, we can reduce the (very difficult)
problem of satisfaction for first order logic to checking a judgement in the calculus, which
indicates type checking \textsc{Refine} is NP-hard (and limited by our best efforts to automate first order proofs).

To be able to check types, I break the problem down into logic which can be dealt with by an SMT
solver using a method I discuss in the next chapter.

\subsection{Another Calculus}

While researching calculi related to refinement types, I found another calculus which I nearly
settled on for my implementation.
This paper by Aguirre \cite{aguirre16} describes a calculus, and includes a mapping to SMT which
is proved correct.
In comparison, Denney uses a more complex calculus, but isn't at all concerned with implementing
an algorithm to check his type system.
While I didn't use his calculus, I drew some ideas from Aguirre such as using an apply (\texttt{@})
function to represent function application to deal with higher order functions.
Aguirre's calculus lacked dependent typing and records, and only had naturals as a ground type,
because of these limitations and because Aguirre's calculus already had a translation to SMT, I
chose Denney's calculus.
