\section{Type Checker Theory}

\subsection{From the introduction}

\textit{This needs to be woven into this chapter properly}

I leave out one of Denney's typing rules ($\lambda$-$\beta$-EQ) as it is hard for an SMT solver to reason with and
the same expressiveness can be achieved without it by choosing a more restrictive $\phi$.
I also define the sets of ground types (Table \ref{tab:ground_types}), constants (Table \ref{tab:constants}),
predicates (Table \ref{tab:predicates}) and axioms (Figure \ref{fig:axiom_schemas}) that Denney uses in his
calculus.
My implementation of $u8rec$ is based on Denney's for $natrec$ but I leave out 2 typing rules as Z3 isn't
capable of inductive reasoning which is required to allow it to work with them.
Fortunately, again by carefully selecting the type guard on the iterator function argument to $u8rec$,
we don't lose expressiveness without these axioms, but we essentially force the programmer to specify the
induction so it can be abstracted out before the SMT solver is run.

\begin{table}
    \centering
    \begin{tabular}{|l|l|}
        \cline{1-2}
        $bool$ & Booleans\\\cline{1-2}
        $u8$ & Unsigned 8 bit integers\\\cline{1-2}
    \end{tabular}
    \caption{Ground Types}
    \label{tab:ground_types}
\end{table}

\begin{table}
    \centering
    \begin{tabular}{|l|l|p{0.508\textwidth}|}
        \cline{1-3}
        $true$, $false$ & $bool$ & Literals for the booleans.\\\cline{1-3}
        $\land$, $\lor$, $\Rightarrow$, $\Leftrightarrow$ & $bool, bool \rightarrow bool$ & Standard binary boolean operators.\\\cline{1-3}
        $\lnot$ & $bool \rightarrow bool$ & Unary not.\\\cline{1-3}
        $0$, ..., $255$ & $u8$ & Literals for the 8 bit unsigned integers.\\\cline{1-3}
        $succ$, $pred$ & $u8 \rightarrow u8$ & The successor and predecessor of a $u8$, which will over/underflow.\\\cline{1-3}
        $+$, $-$ & $u8, u8 \rightarrow u8$ & Addition and subtraction for $u8$, which will over/underflow.\\\cline{1-3}
        $u8rec$ & $\tau, (u8 \rightarrow \tau \rightarrow \tau), u8 \rightarrow \tau$ &
        Recurse over the $u8$s less than a given $u8$ in ascending order.\newline
        $u8rec\ t\ t'\ 0 \coloneqq t$ \newline
        $u8rec\ t\ t'\ succ(n) \coloneqq t'\ n\ (u8rec\ t\ t'\ n)$\\\cline{1-3}
    \end{tabular}
    \caption{Constants}
    \label{tab:constants}
\end{table}

\begin{table}
    \centering
    \begin{tabular}{|l|l|l|}
        \cline{1-3}
        $bool$ & $bool$ & Interpret a boolean expression as a proposition.\\\cline{1-3}
        $\textgreater$, $\textless$, $\leq$, $\geq$ & $u8, u8$ & Integer inequalities.\\\cline{1-3}
    \end{tabular}
    \caption{Predicates}
    \label{tab:predicates}
\end{table}

\begin{figure}
    \centering
    \begin{prooftree}
        \hypo{\Gamma \vdash n: u8}
        \hypo{\Gamma \vdash t: \phi[0]}
        \hypo{\Gamma \vdash t': \Pi_{n:nat} \Pi_{y:\phi[n]} \phi[succ(n)]}
        \infer3[(U8REC)]{\Gamma \vdash u8rec\ t\ t'\ n: \phi[n]}
    \end{prooftree}
    \caption{Axiom Schemas}
    \label{fig:axiom_schemas}
\end{figure}

\subsection{Introducing The Calculus}

The type checker is easily the most complex aspect of the compiler, as the types it checks are capable
of expressing so much more than those of most languages.
Denney's calculus, which mine is based on, is far more complex than the simply typed lambda calculus
even though it only introduces tuples and refinement types.
To give a sense of the difference, lets compare deriving types for a term
with (Figure \ref{fig:with_refinements}) and without (Figure \ref{fig:without_refinements})
refinements.
Note that in the refinement typed example I use the type $4$ to mean $(x:u8) x =_{u8} 4$.

\begin{figure}
    \centering
    \begin{prooftree}
        \infer0{x:4 \vdash x: (x:u8)x =_{u8} 4}
        \infer0{x:4, x:u8, x =_{u8} 4 \vdash x:u8}
        \infer1{x:4 \vdash u8 \sqsubseteq 4}
        \infer2{x:4 \vdash x: u8}
        \infer0{x:4 \vdash x:4}
        \infer1{x:4 \vdash x =_{u8} 4}
        \infer1{x:4 \vdash x + 2 =_{u8} 6}
        \infer2{x:4 \vdash x: (y:u8) y+2 =_{u8} 6}
    \end{prooftree}
    \caption{With Refinement Types}
    \label{fig:with_refinements}
\end{figure}

\begin{figure}
    \centering
    \begin{prooftree}
        \infer0{x:u8 \vdash x:u8}
    \end{prooftree}
    \caption{Without Refinement Types}
    \label{fig:without_refinements}
\end{figure}

It is clear to see how much of a difference the refinement types make to the complexity of type checking.
Denney's calculus has 62 typing rules, which don't include the extra rules that would need to be added
for ground types like $bool$ and $u8$ to allow them to be useful in refinement types.

\subsection{Implementing The Typing Rules}

Type checking for a standard language can be done using the typing rules of the calculus for the language.
This is because the bottom judgement in a typing rule is unique to that typing rule, and if a judgement
fits the bottom of a rule, the requirements for the top rule can be generated from it.
For example, suppose we wish to show the typing $\langle \rangle \vdash \lambda x: u8 . x: u8 \rightarrow u8$
in the simply typed lambda calculus.
We try to construct a proof tree from the bottom up.
The bottom rule must be (ABS), which must require $x:u8 \vdash x : u8$.
Again only one rule can be used which is (VAR) to complete the tree.
This way we can easily construct a proof tree if it exists, or prove it doesn't exist for any judgement.
Now we try the same exercise using Denney's calculus, supposing we want to prove that
$\langle \rangle \vdash \lambda x: u8 . x : u8$ cannot have a proof tree.
The rules that could be at the bottom of our tree are (EQ) or (WEAKEN).
Both of these have a part of the top of the rule that isn't mentioned in the bottom.
To prove we can't use (EQ) we have to show that there exists no $t'$ such that
$\langle \rangle \vdash \lambda x: u8 . x =_{u8} t'$, which is extremely difficult.
To prove we can't use (WEAKEN) we have to show that there exists no $\phi'$ such that
$\langle \rangle \vdash \lambda x: u8 . x : \phi'$ and $\langle \rangle \vdash \phi \sqsubseteq \phi'$
which is very difficult also.
In fact, because of the first order logic embedded into Denney's calculus, we can reduce the
problem of satisfication for first order logic to checking a judgement in the calculus, which
further indicates the difficulty of type checking this language.

To be able to check types, I break the problem down into logic which can be dealt with by an SMT
solver using a method I discuss in the next chapter.

\textit{I plan to add a short discussion about one of the other calculuses I looked at while deciding what to implement to this chapter}

% Other stuff I read about refinement types while researching this
