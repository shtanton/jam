\section{Type Checker Implementation}

\subsection{Implementation Overview}

After reading various papers describing how other similarly typed languages go about type checking,
(\textit{references go here})
and after trying to implement Denney's typing rules directly, I realised that the correct way to
proceed would be to move as much functionality as possible to an SMT solver.
My implementation draws a lot of inspiration from the implementation of a similar calculus by
Aguirre \cite{aguirre16}.
I'll use an example in this chapter to explain how the translation works, the example I'll be
trying to prove is $let\ n: (n:u8) n > 3 = 5\ in\ n$.
After the syntactic sugar is removed, we are left with the program
$(\lambda n: (n:u8) n > 3 . n)\ 5$.

\subsection{Find Type Guards}

First we look at where type guards are applied, as these are the areas where we need to check
that terms satisfy certain conditions.
There are 5 ways in which a type guard can be applied, but 4 of them don't involve refinement types.

\begin{itemize}
    \item Applying a function defined by the user.
    The function has a type guard on arguments passed to it which is defined by the programmer.
    This is the only type guard that can have refinement types.
    \item Using $\pi_i$ to extract a value from a 2-tuple.
    We need to check that the term being extracted from is in fact a 2-tuple.
    \item Calling a constant.
    Arguments to constants in my language are restricted to one of the unrefined types.
    \item Calling a predicate.
    Similarly, arguments to predicates are restricted to one of the unrefined types.
    \item The whole program must be of type $u8$.
    This will be the exit code once the program terminates.
    Further development would probably lead to changing the program to be of a more versatile type,
    perhaps using monads to enable IO and other side effects.
\end{itemize}

The type guards that don't use refinement types actually function identically to the
$\lambda^{\times \rightarrow}$ calculus, so I type check those while doing the labelling pass.
This pass also gives every variable a unique id which simplifies things later on, so our example
becomes $(\lambda n: (m: u8) m > 3 . n)\ 5$.
Type checking calls to user defined functions is more challenging as they make use of refinement
types, and the first step is to find all instances of these so the compiler can check them individually.
For this I wrote a function \texttt{find\_applications} which traverses the labelled syntax tree
and finds applications.
This finds the terms used in the application, and the context around the application, but the compiler
needs the type of the parameter of the function to be able to check the argument term against it.
Fortunately, this is not impossible.

Type inference for refinement types is very challenging, but we don't need complete type inference,
just the ability to infer the parameter type of a term.
This is still tricky, and I do so using 2 utility functions \texttt{arg\_type} and \texttt{arg\_type\_of\_type}.

\subsubsection{arg\_type}

\texttt{arg\_type} takes 3 arguments in order to determine the type of the parameter of an expression.

\begin{itemize}
    \item An expression, which it will try to find the parameter type of, assuming it has undergone
    some transformations.
    \item The context to allow it to get the types of variables it needs.
    \item A sequence of transformations, each of which could be a $\pi_1$, $\pi_2$ or an application.
\end{itemize}

\texttt{arg\_type} answers the question

\begin{quote}
    If I apply a series of transformations to an expression, what will be the parameter type of the
    result?
\end{quote}

At the centre of the expression will be either a variable, with a type we know from context, or an
abstraction, with a parameter type defined by the user.
The function must traverse down through the tree of the expression finding one of these, and as it
does building a stack of the operations it needs to apply on its way back up.
Then as it comes back up, it pops elements off the stack to finally reach the parameter type of the
original expression.
Suppose in the context $f: \Sigma_{m:u8} \Pi_{n:(n:u8) n > m} \Pi_{x:(x:u8) x > m \land x < n} u8$
we see the application $(\pi_2(f)\ 5)\ 4$.
\texttt{arg\_type} would work as shown in Table \ref{tab:arg_type} to generate a type guard of
$4: (x: u8) x > \pi_1(f) \land x < 5$.

\begin{table}
    \centering
    \begin{tabular}{|l|l|l|}
        \hline
        \textbf{Function} & \textbf{Expression} & \textbf{Transformations}\\\hline
        \texttt{arg\_type} & $\pi_2(f)\ 5$ & $nil$\\\hline
        \texttt{arg\_type} & $\pi_2(f)$ & $Arg(5)$\\\hline
        \texttt{arg\_type} & $f$ & $Second(f), Arg(5)$\\\hline
        \multicolumn{3}{|c|}{
            \textbf{Returns}
        }\\\hline
        \multicolumn{3}{|l|}{
            \texttt{arg\_type\_of\_type(}
            $\Sigma_{m:u8} \Pi_{n:(n:u8) n > m} \Pi_{x:(x:u8) x > m \land x < n} u8$, \quad
            $Second(f), Arg(5)$
            \texttt{)}
        }\\\hline
        \multicolumn{3}{|l|}{
            \texttt{arg\_type\_of\_type(}
            $\Pi_{n:(n:u8) n > m} \Pi_{x:(x:u8) x > m \land x < n} u8$, \quad
            $Arg(5)$
            \texttt{)}
            $[\pi_1(f) / m]$
        }\\\hline
        \multicolumn{3}{|l|}{
            \texttt{arg\_type\_of\_type(}
            $\Pi_{x:(x:u8) x > m \land x < 5} u8$, \quad
            $nil$
            \texttt{)}
            $[\pi_1(f) / m]$
        }\\\hline
        \multicolumn{3}{|l|}{
            $(x:u8) x > m \land x < 5$ \quad
            $[\pi_1(f) / m]$
        }\\\hline
        \multicolumn{3}{|l|}{
            $(x:u8) x > \pi_1(f) \land x < 5$
        }\\\hline
    \end{tabular}
    \caption{arg\_type running}
    \label{tab:arg_type}
\end{table}

Running these algorithms on the example program of $(\lambda n : (n: u8) n > 3 . n)\ 5$ produces the
type guards in Table \ref{tab:example_guards}

\begin{table}
    \centering
    \begin{tabular}{|l|l|l|}
        \hline
        \textbf{Context} & \textbf{Term} & \textbf{Type}\\\hline
        $nil$ & $(\lambda n: (n: u8) n > 3 . n)\ 5$ & $u8$\\\hline
        $nil$ & $5$ & $(n: u8) n > 3$\\\hline
        $n: u8$ & $n$ & $u8$\\\hline
        $n: u8$ & $3$ & $u8$\\\hline
    \end{tabular}
    \caption{Type Guards For The Example Program}
    \label{tab:example_guards}
\end{table}

We have now reduced our goal to something close to what an SMT solver can do:

\begin{quote}
    Find an environment that satisfies the context such that the term is not of the required type.
\end{quote}

\subsection{Lambda Lifting}

Unfortunately, there is still an obstacle preventing an easy translation to SMT.
To ask an SMT solver to find such an environment, it needs to understand the term to be able to reason
about it, thus we need to be able to embed the term into the SMT program.
This is almost easy, but Denney's calculus utilises higher order functions, which are not within
the capabilities of first order logic, nor the Z3 SMT solver.

\subsubsection{Defunctionalization}

As an aside, my first thought for how to do this was to defunctionalize the program, which I later
refined down to an algorithm less complex than defunctionalization.
I implemented this algorithm, which first lambda lifted abstractions, then removed tuples from the
program (by introducing 2 separate variables whenever a tuple variable was introduced), then removed
all higher order functions using uncurrying and specialization.
This worked nearly perfectly, but tragically failed if a variable of function type was introduced
in a universal quantifier, as defunctionalization and specialization both rely on knowing a finite
set of functions used in the same way, but if a higher order function comes from a universal
quantifier, then this set is infinite so a function receiving it as an argument cannot be specialized.
This led to me losing dozens of hours of work as I only realised this problem after implementing
almost all of the type checker this way.

\subsubsection{Lambda Lifting}

Because SMT solvers don't have lambda abstractions, we need to change the way our functions are defined
to make use of what they do have: named global functions.
There is a simple process to translate the language into this, called lambda lifting, which I learned
about here \cite{barbosaROTB19}.
For any abstraction, we define a new function $f$ which has a parameter for each free variable in the
abstraction, as well as the original parameter of the abstraction.
We translate the abstraction to a partial application of $f$ to the required environment variables,
which gives us an identical expression.
Dependent typing of functions allows the compiler to add the function to the context instead of
creating a new list of functions.
Looking at Table \ref{tab:example_guards} we can see that only the first contains an abstraction,
which we will move into the context by defining $f$ to be a function with a type that forces it to
be equivalent to the abstraction.
Table \ref{tab:example_lambda_lifted} shows the lambda lifted program type guards.

\begin{table}
    \centering
    \begin{tabular}{|l|l|l|}
        \hline
        \textbf{Context} & \textbf{Term} & \textbf{Type}\\\hline
        $f: \Pi_{n: (n:u8) n > 3} (m: u8) m =_{u8} n$ & $f\ 5$ & $u8$\\\hline
        $nil$ & $5$ & $(n: u8) n > 3$\\\hline
        $n: u8$ & $n$ & $u8$\\\hline
        $n: u8$ & $3$ & $u8$\\\hline
    \end{tabular}
    \caption{Lambda Lifted Type Guards For The Example Program}
    \label{tab:example_lambda_lifted}
\end{table}

\subsubsection{Higher Order Functions}

The best method to simulate higher order functions in an SMT solver seems to be defining a polymorphic
apply function, which takes as arguments a function to apply and an argument to apply it to and returns
the result of the application.
This is discussed everywhere I could find refinement type checking
\cite{aguirre16}, \cite{barbosaROTB19}, \cite{mengP08}, \cite{blanchetteBP13}.
Each function type is represented by an uninterpreted type in Z3 which allows the SMT solver to use as
many different values in the type it needs to find a solution.
There is no definition that accompanies the apply functions either, but we give Z3 some information
about them each time a function is defined.
For example, to define the function in the example program, the SMT-LIB2 code looks like this.

\begin{alltt}
    (declare-const f fun-u8-to-u8)
    (assert (forall ((n u8)) (=> (> n 3) (= (@ f n) n))))
\end{alltt}

Which means that for any $u8$ which is greater than 3, applying $f$ to it is equal to the original value.
Using this means that user defined functions are just values which the SMT solver can work with
easily, so higher order functions are no longer a problem.

\subsection{Translation To SMT}

I've already shown how I encode application to circumvent the lack of higher order functions in Z3.
To make the code more readable I split the translation up into 2 parts.
The first produces a program encoded using structures I defined in Rust, which makes it much easier
to work with while doing the translation.
I then map that onto an actual Z3 program using the Rust Z3 library which takes very little effort
due to my Rust data structures already being very close to how programs are encoded in Z3.

\subsubsection{My SMT Structure}

My SMT data structure consists of 3 lists:

\begin{itemize}
    \item \textbf{Constant Declarations} -
    A list of variables from the context, each with an \textit{unrefined type}.
    The SMT solver will be tasked with finding values for these variables to reject a type guard.
    We have to use unrefined types as these are what the SMT solver can understand, but we force
    Z3 to find values that are acceptable for the refined types by adding an assertion for each
    variable in the context, restricting it.
    \item \textbf{Assertions} -
    A list of expressions with type $bool$.
    We task Z3 with finding a value for each constant such that every assertion evaluates to $true$.
    There is an assertion restricting each constant to its refined type, as well as one which asserts
    that the expression we are checking has the type required by the guard.
    Since we are looking for an exception, the last assertion asserts that the expression does not
    satisfy the refinement on its type, thus if Z3 concludes with unsat then every combination must
    produce a value for the expression that is acceptable to the type.
    \item \textbf{Unrefined Types} -
    A list of unrefined types that are used in the program, ordered so any part of a type appears
    before the type itself does.
    This is generated by maintaining a list as the judgement is traversed, and whenever a type is
    found, calling a \texttt{register\_type} function to add its parts and then it to the list if
    they aren't already present.
\end{itemize}

Note that an unrefined type is a type that contains no refinements or dependent typing, i.e. a type
from the $\lambda^{\times \rightarrow}$ calculus.

In order to generate the assertions, I need to split a type into an unrefined type, and a single
predicate which refines it into the actual type.
I do this using a \texttt{simplify} function that passes \textbf{1}, $bool$ and $u8$ through,
but traverses down 2-tuple, functions and refined types to combine all of the refinements into one
predicate.
Denney conveniently showed how to represent 2-tuple and function types as refined types in his paper
by interpreting

\begin{center}
    $\Sigma_{x:(x:\sigma)P} (y:\tau)Q$
    \quad as \quad
    $(z: \sigma \times \tau)P[\pi_1(z)/x] \land Q[\pi_1(z)/x, \pi_2(z)/y]$

    and

    $\Pi_{x:(x:\sigma)P} (y:\tau)Q$
    \quad as \quad
    $(f: \sigma \rightarrow \tau) \forall x : \sigma . P \supset Q[f\ x / y]$
\end{center}

With these representations in mind, we take the conjunction of all the refinements applied to a type
to get a predicate that covers all of them.
Once we have this, it becomes straightforward to generate all the assertions for the SMT program.

Continuing the example, Table \ref{tab:example_smt} shows the SMT programs generated for each
judgement.
\begin{table}
    \centering
    \begin{tabular}{|l|l|l|}
        \hline
        \textbf{Context} & \textbf{Term} & \textbf{Type}\\\hline
        $f: \Pi_{n: (n:u8) n > 3} (m: u8) m =_{u8} n$ & $f\ 5$ & $u8$\\\hline
        \multicolumn{3}{|p{1.0\textwidth}|}{
            \texttt{(declare-sort u8\_to\_u8)}\newline
            \texttt{(declare-fun apply-u8\_to\_u8 (u8\_to\_u8 u8) u8)}\newline
            \texttt{(declare-const f u8\_to\_u8)}\newline
            \texttt{(assert (forall ((n u8)) (=> (> n 3) (= (apply-u8\_to\_u8 f n) n))))}\newline
            \texttt{(assert (not true))}
        }\\\hline
        $nil$ & $5$ & $(n: u8) n > 3$\\\hline
        \multicolumn{3}{|p{1.0\textwidth}|}{
            \texttt{(assert (not (> 5 3)))}
        }\\\hline
        $n: u8$ & $n$ & $u8$\\\hline
        \multicolumn{3}{|p{1.0\textwidth}|}{
            \texttt{(declare-const n u8)}\newline
            \texttt{(assert true)}\newline
            \texttt{(assert (not true))}
        }\\\hline
        $n: u8$ & $3$ & $u8$\\\hline
        \multicolumn{3}{|p{1.0\textwidth}|}{
            \texttt{(declare-const n u8)}\newline
            \texttt{(assert true)}\newline
            \texttt{(assert (not true))}
        }\\\hline
    \end{tabular}
    \caption{SMT Programs For The Example}
    \label{tab:example_smt}
\end{table}
A lot of the assertions in Table \ref{tab:example_smt} seem trivial.
This is because for a judgement like $3:u8$, the unrefined type of $3$ has already been checked in
the labelling stage, and there are no refinements on the type so there is no more to check.
These extra assertions make little difference to the performance of the solver, so I leave them
in rather than finding a way to avoid them.

\subsubsection{Running With z3.rs}

This translation is made very simple as the previous step has already got very close to the format
expected by Z3.
Translating the types is the most interesting part, as I need to simulate some polymorphism for
the apply function and the functions for creating and extracting from 2-tuples.
The types \textbf{1}, $bool$ and $u8$ are easy, and type refinements were removed in the last
stage so only tuples and functions are left.
Each function type becomes an uninterpreted type and each has an associated apply function for
applying functions of that type.
Each tuple type is used to define a tuple in Z3, with functions for creating and extracting from
them.
The function types and tuple types are stored in maps to allow the compiler to access the type
representations or functions when needed while translating the assertions and declarations.
The compiler then constructs the program from the assertions and declarations and runs Z3.
If Z3 returns unsat, then there is no environment that satisfies the context such that the value
of the term doesn't satisfy the requirements of the type, thus the term can be typed.
If Z3 returns sat, then there is a typing error in the program, which the compiler reports to the
user.
One of the advantages of this method is that it provides a counterexample when a type check fails,
and if I had more time I would like to have added the ability for the compiler to present the
counterexample to the user in a readable way to make debugging the error much easier.
As things stand, the compiler gives the counterexample exactly as it gets it from Z3, which is less
readable than I would like, but still very useful.

\subsection{Implementing u8rec}

Implementing $u8rec$ based on Denney's $natrec$ was an interesting challenge, due to the limitations
of the SMT solver.

\subsubsection{Implementing u8rec in Z3}

The obvious way to implement u8rec would be to define a \texttt{u8rec} function in Z3, which can be
used in translated expressions.
We could do so using 2 of Denney's axiom schemas which describe equalities.

\begin{alltt}
(declare-sort bool_to_bool)
(declare-fun apply (bool_to_bool bool) bool)
(declare-sort bool_iter)
(declare-fun apply_bool_iter (bool_iter u8) bool_bool)
(define-fun succ ((x u8)) u8
  (bvadd x 1)
)
(declare-fun u8rec_bool (bool bool_iter u8) bool)
(assert (forall ((x bool) (x_iter bool_iter)) (=
  (u8rec_bool x x_iter 0)
  x
)))
(assert (forall ((x bool) (x_iter bool_iter) (n u8)) (=>
  (not (= n 255))
  (=
    (u8rec_bool x x_iter (succ n))
    (apply_bool_bool (apply_bool_iter x_iter n) (u8rec_bool x x_iter n))
  )
)))
\end{alltt}

This makes it very easy to encode uses of $u8rec$ as they can be translated directly, however, when
we try running Z3 on a program like this we encounter a problem, it runs for a while and then returns
unknown.
This is because Z3 cannot prove inductive facts (\textit{Do I need a citation for this?}) which is
required to prove any property making use of both of these rules.

\subsubsection{Type Focused Implementation Of u8rec}

Due to this limitation, I found that I could leave out 2 of Denney's axiom schemas, and use only the
remaining one to be accessible to Z3, without greatly limiting its usefulness.
The rule I keep is

\begin{center}
    \begin{prooftree}
        \hypo{\Gamma \vdash n : u8}
        \hypo{\Gamma \vdash t : \phi[0]}
        \hypo{\Gamma \vdash t' : \Pi_{n:(n:u8) n \neq_{u8} 255} \Pi_{p:\phi[n]} \phi[succ(n)]}
        \infer3[(U8REC)]{\Gamma \vdash u8rec\ t\ t'\ n : \phi[n]}
    \end{prooftree}
\end{center}

The compiler is able to reason with this rule by determining in the labelling pass what the type
$\phi[n]$ is, and then it adds 3 type judgements based on $\phi[n]$ for $n$, $t$ and $t'$ and
replaces the occurence of $u8rec\ t\ t'\ n$ with a variable of type $\phi[n]$.
This means that after the lambda lifting pass, all occurences of $u8rec$ are gone so the translation
to SMT doesn't need to be able to deal with them.
This approach effectively forces the user to provide the base case and inductive hypothesis explicitly,
as the compiler gets the definition of $\phi[n]$ from the term $t'$ using the previously mentioned
utility function \texttt{arg\_type}.

To see that we don't lose the power of $u8rec$ when we lose these axioms, consider for example that
we wish to show that some $u8rec$ term has a value $t$.
We could expand on this problem to try to show that $u8rec\ t_0\ t'\ n = t[n]$.
Then we simply rewrite this as $u8rec\ t_0\ t'\ n: (x: \phi)x = t$ to enable us to prove it using
the one (U8REC) rule.
In the most extreme case, this requires the user to use a type which can only have one possible value
assigned to it, so is less convenient, and might take more steps to prove the typing of the subterm.

Similarly, I don't use Denney's rule ($\lambda$-$\beta$-EQ) as it allows circumventing type guards
in some instances.
This would make life easier for the user as they would be able to use less descriptive types for
parameters while still making strong assumptions about them as long as any arguments used have
acceptable types, but it would add too much complexity to the compiler and is only more useful
for saving the user time rather than improving the functionality of the language.
As with the earlier example, we could instead use (ABS) and (APP) with more restrictive types,
which has stronger requirements on the types involved, using a proof tree such as in Figure
\ref{fig:beta_eq}.

\begin{figure}
    \centering
    \begin{prooftree}
        \hypo{\textit{dependent on structure of t}}
        \infer1{x:(x:\phi)x=t' \vdash t = t[t'/x]}
        \infer1{x:(x:\phi)x=t' \vdash t:(y:\psi)y=t[t'/x]}
        \infer1{\lambda x: (x:\phi)x=t' . t : \Pi_{x: (x:\phi)x=t'} (y:\psi)y=t[t'/x]}
        \infer0{t': (x:\phi)x=t'}
        \infer2{(\lambda x: (x:\phi)x=t' . t) t' : (y:\psi)y=t[t'/x]}
        \infer1{(\lambda x: (x:\phi)x=t' . t) t' = t[t'/x]}
    \end{prooftree}
    \caption{An Alternative To ($\lambda$-$\beta$-EQ)}
    \label{fig:beta_eq}
\end{figure}

\subsection{Optimisation}

My compiler has a lot of room for optimisation, but it was not a focus for me, as the role of my
compiler is much simpler than that of Z3.
Running the SMT solver will always be the most time consuming part of the type checking.
There are various computations such as computing the unrefined type of an expression after the lambda
abstraction stage, which are redone many times and would benefit from being cached.
Also once split into judgements, they are checked independently and in sequence, even though there
is a lot of overlap as they all come from the same program.
It would be possible to do a lot of the computation before splitting the program into judgements,
and then checking them all in parallel to speed up type checking.
The biggest change I could make, would be to make changes to the SMT program that is produced to
make it quicker to check for satisfication.
Because of the equivalence

\begin{center}
    $\lnot \forall x \in X . P \equiv \exists x \in X . \lnot P$
\end{center}

we can expand universal quantifiers in the final assertion (which is always a negation) to additional
constants, and so remove instances of quantification, which Z3 is not as good at dealing with.
