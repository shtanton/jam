\section{Introduction}

I have developed a compiler for a language I call Refine,
which has runtime semantics much like Haskell, but typing semantics much like a calculus defined
by Denney \cite{denney98}, which I'll call $\lambda^{\times \rightarrow \sqsubseteq}$.
$\lambda^{\times \rightarrow \sqsubseteq}$ and Refine both make use of refinement types to push
the boundaries formal verification.
This means properties of a program can be formally verified, mathematically proving them to be true
and improving the reliability of software.

\subsection{What Are Refinement Types?}

A recent trend for programming languages has been improving the compiler's ability to find
bugs in programs at compile time to improve the robustness of compiled code.
Building formal verification tools into compilers can be done in many ways, but the method
I've chosen to explore is using \textit{refinement types}.
In any programming language, a type can be considered to be a set of all the possible values
that a variable of that type could have.
For example, the boolean type could be considered as the set $\{true, false\}$.
Refinement types allow the programmer to add \textit{refinements} to these types which
filter the set using a predicate.
Using Denney's syntax for a refinement type, the type
$(x:nat)x<5$ corresponds to the set $\{x \in nat | x < 5\}$ which is equivalent to
$\{0, 1, 2, 3, 4\}$.
Because Refine strongly resembles the $\lambda^{\times \rightarrow}$ calculus
with simple types, it is easy to see that any term can be fully $\beta$ and $\eta$ reduced to
arrive at a value for the term.
If, in a given context, every possible value a term can take is in the set of values corresponding to
a type, then the term can have that type.

\subsection{Why Are Refinement Types Useful?}

A crucial part of any software is reliability and robustness.
Currently, the most widely used strategy for ensuring that software works as it should
is to run automated tests.
\begin{quote}
    Program testing can be a very effective way to show the presence of bugs, but is
    hopelessly inadequate for showing their absence.
    The only effective way to raise the confidence level of a program significantly is
    to give a convincing proof of its correctness. - Edsger W. Dijkstra \cite{dijkstra72}
\end{quote}
This shows a flaw with running tests, they cannot possibly check every possible set of inputs
to the program, so cannot give any guarantees outside of the test cases that are checked.
If there is an edge case for which a function doesn't perform correctly which is overlooked
and so not tested, it can easily make it into production where, depending on where it is
deployed, it could have disastrous consequences.
Formal verification can prove properties formally for every possible sequence of inputs.
This solves the problem of missing certain edge cases, but doesn't solve the problem of
needing to provide the correct property to check.
Frequently the property that is being checked matches the implementation, for example,
a successor function $succ$ would have a type like $\Pi_{x:nat} (y:nat) y =_{nat} x+1$
and an implementation like $\lambda x:nat . x+1$.
Clearly the refinement type isn't adding any value here,
but there are classes of problems for which refinement types are very useful.

The first is a problem where verifying a correct solution is much easier than finding
the solution.
Denney uses the example of $div2$ which divides a natural number by 2, rounding down.
Finding the solution to $div2\ n$ involves iterating $n$ times and alternately incrementing
2 counters, which is far more complex than doubling the result and seeing if it is $n$ or
$n-1$.
\begin{quote}
    If verifying a result is simpler than computing a result then refinement types can prove that
    every result is correct.
\end{quote}

The second is a problem where there are 2 algorithms which could be used to compute a value.
One has preferable runtime characteristics (such as better time or space complexity), while
the other is easier to prove by hand.
Refinement types allow the programmer to implement both, then have the compiler check they are
equivalent to prove that the more useful implementation also functions correctly.
Suppose we want to check if a natural number, $n$, is even.
We already have a $div2$ function which we know works, so we could run it, then double the
result and compare it to $n$.
It is easy to prove that $n$ is even iff $double\ (div2\ n)$ is equal to $n$, but this algorithm
is very convoluted, and doing this much work is unnecessary.
Instead we could opt for another, simpler algorithm, and have the compiler prove that it functions
identically to the first algorithm.
The compiler would prove this every time it is run, so if we change our implementation later on,
we don't need to worry that we've broken something.
\begin{quote}
    If it is hard to prove an efficient algorithm works, prove an inefficient algorithm and let the
    compiler prove they're equivalent.
\end{quote}

There may be other cases for which refinement types can be particularly useful, but at least in
these 2 instances, a refinement typed language can give much more confidence than testing ever
could.

\begin{itshape}
    I'll add an example of the type checker finding a non-obvious bug in some code here once the
    type checker is more capable.
\end{itshape}

\subsection{Refine}

Refine is based on $\lambda^{\times \rightarrow \sqsubseteq}$, with a few changes.
$\lambda^{\times \rightarrow \sqsubseteq}$ has 5 types (Table \ref{tab:types}),
7 terms (Table \ref{tab:terms}) and propositions, which are mostly first order logic constructs
that I won't repeat, though I show the most interesting 2 in Table \ref{tab:propositions}.

\begin{table}
    \begin{tabular}{|r|p{0.8\textwidth}|}
        \hline
        \textbf{1} &
        The unit type, a term of type \textbf{1} must have the value $\ast$.\\\cline{1-2}
        $\gamma$ &
        A ground type, in Refine this is one of $bool$ (boolean) or $u8$ (unsigned 8 bit integer)\\\hline
        $\Sigma_{x:\phi} \psi$ &
        A pair of values (2-tuple).
        The first has type $\phi$ and the second has type $\psi$.
        Contrary to the $\lambda^{\times \rightarrow}$ calculus, the type of the second value
        can be dependent on the first value.
        For example a tuple of 2 numbers could be restricted so the second is equal to the
        first with the type
        $\Sigma_{x: u8} (y: u8) x =_{u8} y$.
        We refer to the ability for a type to depend on a value as \textit{dependent typing}\\\cline{1-2}
        $\Pi_{x:\phi} \psi$ &
        A function from $\phi$ to $\psi$.
        These are also dependently typed, the type of the return value can be restricted based on the
        value of the argument.
        Any argument used to call the function should have type $\phi$, this is how we impose
        \textit{type guarding}.\\\cline{1-2}
        $(x:\phi)P$ &
        A type $\phi$ which is \textit{refined} by filtering the possible values of a term using the
        predicate $P$.
        For example, were the type $(x: u8) x < 3$ used on a term, then at runtime the value of the
        term must but in the set $\{0, 1, 2\}$.\\\hline
    \end{tabular}
    \caption{Types}
    \label{tab:types}
\end{table}

\begin{table}
    \begin{tabular}{|r|p{0.8\textwidth}|}
        \hline
        $x$ &
        A variable, which will be bound as the parameter of a function or from dependent typing.\\\cline{1-2}
        $k(t_1, ..., t_n)$ &
        A constant, which takes $n$ arguments and returns a value.
        Unlike a function value defined by the user, constants can have axioms that govern how the typechecker
        should reason about them.\\\cline{1-2}
        $\ast$ &
        The value of type \textbf{1}\\\cline{1-2}
        $\langle t, t' \rangle$ &
        Creates a 2-tuple value.\\\cline{1-2}
        $\lambda x: \phi . t$ &
        Creates a function value.
        $t$ is considered in a context where $x$ has type $\phi$.\\\cline{1-2}
        $\pi_i(t)$ &
        Extract a value from a 2-tuple. $i \in \{1, 2\}$.\\\cline{1-2}
        $t\ t'$ &
        Apply function $t$ to argument $t'$.
        Refine uses call-by-name semantics.\\\hline
    \end{tabular}
    \caption{Terms}
    \label{tab:terms}
\end{table}

\begin{table}
    \begin{tabular}{|r|p{0.8\textwidth}|}
        \hline
        $t =_\phi t'$ &
        This functions almost exactly as you would expect from equality, with the exception of function types.
        2 functions can be equal even if they don't have the same image, we only care about functions
        restricted to the domain of $\phi$.
        Denney's (ABS-$\xi$-EQ) rule is used for this, and allows us to prove propositions like\newline
        $\lambda x: bool . x \quad =_{(x:bool) x =_{bool} true\ \rightarrow\ bool} \quad \lambda x: bool . true$.
        \\\cline{1-2}
        $\phi \sqsubseteq \psi$ &
        This is defined to mean $\phi$ is a supertype of $\psi$,
        so any value which a term of type $\psi$ could have, $\phi$ can also have.\\\hline
    \end{tabular}
    \caption{Propositions}
    \label{tab:propositions}
\end{table}

I've added some extra terms using syntactic sugar to make life easier for the user too.

\begin{center}
    $\lambda x_1: \phi_1, ..., x_n: \phi_n . t \coloneqq \lambda x_1: \phi_1 .\ ...\ \lambda x_n: \phi_n . t$

    $f\ t_1\ ...\ t_n \coloneqq (...(f\ t_1)\ ...)\ t_n$

    $let\ x_1: \phi_1 = t_1, ..., x_n: \phi_n = t_n\ in\ u \coloneqq (\lambda x_1: \phi_1 .\ ...(\lambda x_n: \phi_n . u)\ t_n...)\ t_1$
\end{center}

I define Refine's operational semantics in line with that of the $\lambda^{\times \rightarrow}$ calculus.

\begin{center}
    $(\lambda x: \phi . t) t' \rightarrow_\beta t[t'/x]$

    $\pi_i(\langle t_1, t_2 \rangle) \rightarrow_\beta t_i$
\end{center}

To see the strength of refinement types, consider the simple example of ordering 2 integers in
Figure \ref{fig:refinement_ordering}.
$\Gamma$ is $x: \Sigma_{y:u8} u8$, $c$ is $\pi_1(x) < \pi_2(x)$ and $\phi$ is $\Sigma{y:u8} (z:u8) y \leq z$.

\begin{figure}
    \centering
    \begin{prooftree}[separation=0.5em]
        \infer0{\Gamma, c \vdash \pi_1(x): u8}
        \infer0{\Gamma, c \vdash \pi_2(x): u8}
        \infer0{\Gamma, c \vdash \pi_1(x) < \pi_2(x)}
        \infer1{\Gamma, c \vdash \pi_1(x) \leq \pi_2(x)}
        \infer2{\Gamma, c \vdash \pi_2(x): (z:u8)\pi_1(x) \leq z}
        \infer2{\Gamma, c \vdash \langle \pi_1(x), \pi_2(x) \rangle =_{\phi} x}
        \infer1{\Gamma, c \vdash x: \phi}
        \hypo{Similar}
        \infer1{\Gamma, \lnot c \vdash \langle \pi_2(x), \pi_1(x) \rangle: \phi}
        \infer2{\Gamma \vdash ite\ (\pi_1(x) < \pi_2(x))\ x\ \langle \pi_2(x), \pi_1(x) \rangle: \phi}
    \end{prooftree}
    \caption{Refinement Types For An Ordered Tuple}
    \label{fig:refinement_ordering}
\end{figure}

\subsection{My Compiler}

A compiler for Refine has 2 main jobs.
First, as with any compiler, to produce an executable which can be run to reach an outcome in line
with the semantics of the language.
Secondly, and more interestingly, either prove that any term which is type guarded must evaluate
at runtime to a value acceptable to that type, or find a counterexample to present to the user.
My compiler does both of these jobs using a wide variety of techniques and technologies.

\subsubsection{Rust}

My compiler is implemented in Rust.
I chose Rust as it is a recently developed but stable language, interestingly it also has a focus
on formal verification, but not with refinement types.
It also has a vast library ecosystem and features such as variants and advanced pattern matching
which greatly help with a project like a compiler.
Code written in Rust is responsible for receiving the input program and producing a series of SMT
programs for Z3 and a bytecode program for LLVM.
It also interprets the results from Z3 as they determine if there is a type checking error.

\subsubsection{LLVM}

I chose to compile Refine to LLVM (which Rust coincidentally also compiles to) instead of
directly to any machine code.
LLVM is another relatively recent technology which enables developers to write compilers which can
target a wide array of architectures with much less effort than was historically needed.
It also is highly optimised and when undertaking the project I was aware that I probably wouldn't
have time to optimise the compiler much.
Once Rust has generated the LLVM bytecode, I run the \texttt{llc} and \texttt{gcc} to compile the
bytecode and link it into a standalone executable.

\subsubsection{Z3}

The third big recent technology I made use of was the Z3 SMT solver, which I tasked with formally
proving programs were sound once my type checker had reduced the problem down to first order SAT.
I do this by giving Z3 a set of constraints within which it tries to find a satisfying assignment.
These constraints force it to find a counterexample to the type guards of the Refine program so
either it will find such a counterexample or prove that it doesn't exist.
The largest hurdle is translating a Refine program to a series of SMT programs for Z3.
Z3 lacks various important features like higher order functions and lambda abstractions which I
was forced to find ways to remove without changing the semantics of a term to allow Z3 to understand
said term.
I discuss the difficulties of this problem in much greater detail in the section on
Type Checker Implementation.

\bigskip

Learning how to use each of these technologies was very enjoyable, and tied in with a lot of the
courses I've taken over the last 2 years.
Particularly: compilers, computer aided formal verification, lambda calculus and types, logic and proof
and principles of programming languages all were very useful in enabling me to know how to proceed
with challenging areas of the project.

\subsubsection{Compiler Structure}

The compiler uses a very standard and well tested structure: parsing, labelling, type checking,
IR bytecode generation, linking.
The type checking stage is split further into several stages as it is the most complex due to the
complexity of the type system.
The parsing, labelling and IR bytecode generation stages are not particularly complicated, however
learning to use LLVM for code generation took quite some time and because Refine is functional,
compiling to assembly is not as straightforward as it would be for a procedural language.
