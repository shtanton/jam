\section{The Compiler}

\subsection{Compiler Structure}

The compiler is split into several passes, the first generates an abstract syntax tree, while the others
operate on the tree.
In total, there are 4 passes (not including type checking or linking, as I don't discuss those here).

\begin{itemize}
    \item \textbf{Parsing} -
    Building an AST based on the input string.
    This is a very standard stage common to almost all compilers, as doing later stages on a syntax tree
    is substantially easier than trying to work with a string.
    \item \textbf{Labelling} -
    Replacing string identifiers for variables with globally unique integer identifiers and
    labelling every term with data about which variables it depends on and its type.
    This means the later stages can assume ids are unique, and makes programming environment capture for
    closures or lambda lifting more straightforward.
    \item \textbf{Type Checking} -
    Checking that the labelled AST is valid.
    This doesn't change the AST in any way, but the compiler won't proceed to the next stages if it fails the
    type checks.
    Chapter 2 is focused on the type checker so I don't discuss it any more here.
    \item \textbf{Thunking} -
    This is a tiny pass to produce the call-by-name semantics by replacing arguments with thunks.
    \item \textbf{Code Generation} -
    Write LLVM bytecode for the program to allow it to be executed.
    \item \textbf{Linking} -
    This stage compiles the LLVM IR and links it into a standalone executable.
    I use the tools \texttt{llc} and \texttt{gcc} for this, and didn't write any code myself so I don't go into
    any detail on it.
\end{itemize}

\subsection{Parsing}

I used a Rust library called nom for parsing, which makes use of parser combinators to build the parser.
The idea is to start with simple parsers such as a fixed string and then use combinators like concatenation,
or alternation, or repetition, to construct more complex parsers.
They have a lot in common with regular expressions, but are powerful enough to recognise context free languages
and can be easily extended with new parsers or combinators.

Figure \ref{fig:let_expression_parser} shows the parser for $let$ expressions.
\begin{figure}
    \begin{alltt}
named!(expression_let(&str) -> Expression, do_parse!(
  char!('(') >> ws0 >> tag!("let") >> ws1 >>
  defns: separated_list1!(ws1, definition) >> ws1 >> tag!("in") >> ws1 >>
  e: expression >> ws0 >> char!(')') >>
  (\{
    let mut expr = e;
    for (id, t, value) in defns.into_iter().rev() \{
      expr = Expression::Application(
        Box::new(Expression::Abstraction(id, Box::new(t), Box::new(expr))),
        Box::new(value),
      );
    \}
    expr
  \})
));
    \end{alltt}
    \caption{Let Expression Parser}
    \label{fig:let_expression_parser}
\end{figure}
This and most other parsers use the \texttt{do\_parse} combinator to concatenate several parsers together and then
generate an expression using the values of the individual parsers.
A subparser that will result in a value I intend on using can be labelled (such as \texttt{defns}) and then in
the expression for the result of the parser, that result is available as a variable.
This parser generates the syntactic sugar for the $let...in$ syntax, so each definition is translated into
an immediately applied abstraction.

Figure \ref{fig:expression_parser} shows the parser for any expression.
\begin{figure}
    \begin{alltt}
named!(expression(&str) -> Expression, alt!(
  expression_u8rec |
  expression_let |
  expression_abstraction |
  expression_first |
  expression_second |
  expression_call |
  expression_ast |
  expression_application |
  expression_variable |
  expression_tuple
));
    \end{alltt}
    \caption{Expression Parser}
    \label{fig:expression_parser}
\end{figure}
This uses the \texttt{alt} combinator to try parsers in succession until one of them matches.
Some thought must go into the order of the subparsers as they are tried in order.
If \texttt{expression\_application} were higher up the list, it would be before \texttt{expression\_call} and
other similar looking expressions.
This would mean that any calls to constants were treated as user defined functions being applied, so the
later stages wouldn't work.

\subsection{Labelling}

While traversing the AST, we keep a record of the types of variables, and the ids of variables that are currently
in scope.
This means that each time a variable is referenced, the string identifier can be replaced with the appropriate
unique integer identifier.
Also, the (unrefined) type and free variables used for each term can be used by combining
the information from its subterms.
Having all the extra information easily available makes the later stages of the compiler much simpler,
for example, the type checker and code generator both need to the free variables for each abstraction term.

\subsection{Thunking}

This pass does a number of adjustements, with the goal of replacing a arguments, $v$ with abstractions,
$\lambda x: \textbf{1} . v$.
There are several other changes that need to take place for the syntax tree to remain consistent under this
change.
\begin{itemize}
    \item References to variables, $v$, must be replaced with applications of the thunk $(v\ \ast)$.
    \item Types of parameters in abstractions, $T$, must be replaced with thunks, $\textbf{1} \rightarrow T$.
    \item Types of expressions with function subtypes must have their parameter types changed to thunks.
    \item Types in free variable sets with function subtypes must have their parameter types changed to thunks.
\end{itemize}
This produces an AST which can be evaluated using call-by-value semantics to achieve the same result as
the original AST evaluated with call-by-name semantics.
This transformation is based on what's described in Appel's book \cite{appel1998}.

\subsubsection{Thunking In Code Generation}

Originally I had planned to introduce thunks for arguments in the code generation pass.
While trying to do this, I realised that LLVM IR code for abstractions is challenging, as I need to
implement closures.
I wanted to avoid this complexity where possible, so decided to introduce thunks in a new pass before
code generation, so the code generation for abstractions can be reused to generate the thunks as well.

\subsection{Code Generation}

I chose to use the LLVM toolchain to implement my compiler.
It is a recently developed intermediate representation language, along with a compiler and interpreter,
with the goal of being a target for compilation for new languages, which can then be compiled down to
a wide variety of architectures.
Using LLVM both reduced the amount of work I needed to do, it does for me:
\begin{itemize}
    \item Optimisation -
    Dead code elimination,
    partial evaluation at compile time,
    Loop unrolling,
    Function inlining
    and many other optimisations are done by the LLVM compiler, which somewhat makes up for the fact that
    I have done no optimisation myself.
    \item Structures/Records -
    LLVM has structures as a type, and provides instructions for working with them.
    \item Functions -
    Some instruction sets have functions, but LLVM functions are called similarly to high level functions
    and can take or return large types like structures.
    \item Targeting Several Platforms -
    Had I written my compiler to compile straight to assembly for an architecture, code in Refine could
    only have been executed on that hardware.
    With LLVM my code can be executed on nearly any system.
\end{itemize}
I extensively used the LLVM Language Reference Manual \cite{llvm} to determine which LLVM instructions
would correspond to each of Refine's constructs.
For most this was straightforward, all of my constants other than $u8rec$ have obvious representations
in LLVM, $\ast$ is represented using a null byte, $u8$ and $bool$ are also both bytes with $bool$ using
$0$ and $1$ as $false$ and $true$ respectively and 2-tuples map well onto LLVM structures.

\subsubsection{Closures}

Abstractions (and applications) were particularly challenging as each function needed to capture its
environment, so access to variables in outer scopes was possible from within the function.
I treat a closure as a pointer to a structure on the heap, which has as its first element a pointer
to the function (which takes the closure as its first parameter and the actual parameter as its second),
and its remaining elements are copies of variables it needs from its environment.
This is different to the static link based approach in Appel's book, but is more versatile, as a
static link becomes invalid once the parent function returns and its stack frame is popped.
Because my closures can be returned from functions, a child function might outlive its parent by being
returned and then applied, so it needs a whole copy of the necessary data to prevent any of it being
missing when it is applied.
This has the disadvantage that if any of the data changes since the closure is constructed, it is no longer
correct, however, all state is immutable in Refine so this isn't possible.

Because a closure may outlive any function, I chose to allocate it on the heap, which has the added problem
of cleaning it up by freeing the memory once I'm done with it.
The most sensible way to solve this problem (and the solution I chose) is to use a garbage collector to do this.
There is also the problem that since the closure contains a function which takes as an argument the closure,
the type of the closure is contained within itself so cannot be expressed.
Fortunately, I can use a byte pointer as the type of the closure in the function parameter, and bitcast it
before and after passing it.
Finally there is the problem that 2 closures might have the same type in Refine, but due to differences
in captured variables, have different types in LLVM.
To get around this, once formed the closures are bitcast into a pointer to a struct containing only the
function pointer (to allow them to be called), and then inside the function, they are cast back to
the full environment to access the elements.

An example function written in LLVM bytecode is shown in Figure \ref{fig:closure}.
\begin{figure}
    \begin{alltt}
define \{ i8 (i8*, \{ i8 (i8*, i8)* \}*)* \}* @fn.3(i8* \%0, \{ i8 (i8*, i8)* \}* \%1) \{
fn_block:
1 \%env = bitcast i8* \%0 to
    \{ \{ i8 (i8*, \{ i8 (i8*, i8)* \}*)* \}* (i8*, \{ i8 (i8*, i8)* \}*)* \}*
2 \%malloc = call i8* @GC_malloc(i64 8)
3 \%env1 = bitcast i8* \%malloc to \{ i8 (i8*, \{ i8 (i8*, i8)* \}*)* \}*
4 \%env_func_ptr = getelementptr inbounds \{ i8 (i8*, \{ i8 (i8*, i8)* \}*)* \},
    \{ i8 (i8*, \{ i8 (i8*, i8)* \}*)* \}* \%env1, i32 0, i32 0
5 store i8 (i8*, \{ i8 (i8*, i8)* \}*)* @fn.4,
    i8 (i8*, \{ i8 (i8*, i8)* \}*)** \%env_func_ptr, align 8
6 ret \{ i8 (i8*, \{ i8 (i8*, i8)* \}*)* \}* \%env1
\}
    \end{alltt}
    \caption{Example Function Using A Closure}
    \label{fig:closure}
\end{figure}
Here is a function which will become a closure that takes a $u8$ and returns a $u8 \rightarrow u8$.
Line 1 casts the captured variables from its environment into the appropriate type.
Line 2 allocates space for a closure which is being constructed (8 bytes for a 64 bit pointer).
Line 3 casts the new pointer to the type of the closure.
Lines 4 and 5 store the pointer to the appropriate function in the new closure.
Line 6 returns the closure.
The reason there are so many closures in this simple function, is due to call-by-name using thunks
the type of its parameter and the parameter of the function it returns are both themselves functions.

\subsubsection{Garbage Collection}

Instead of writing a garbage collector myself, I decided on using the Boehm garbage collector.
It is incredibly easy to use, as it just replaces the standard \texttt{malloc} with its own procedure
to allow it to track the memory allocations, and clean them up once no references to them remain.
I only used it for closures, with the rest of the data being stack allocated.

\subsubsection{u8rec}

As with the type checker, implementing $u8rec$ was particularly challenging.
LLVM bytecode uses SSA (static single assignment) form, which means that variables cannot be reassigned to.
Instead of this, we use phi nodes to implement constructs like loops.

A bytecode representation of $u8rec$ has 4 blocks
\begin{itemize}
    \item \textbf{Entry} -
    Constructs a thunk for the initial accumulator value (the value passed through the function each loop).
    \item \textbf{Header} -
    Checks the current value of the counter and jumps to \textbf{Exit} if it is zero or \textbf{Body} if not.
    Because the value of the counter could be either the initial value for the first time the block
    runs, or the value calculated in \textbf{Body} for subsequent runs, I used a phi node do select
    the correct value depending on where this block was jumped to from.
    This produces a term like\newline
    \texttt{phi i8 [ 2, \%u8rec\_entry ], [ \%count, \%u8rec\_body ]}.\newline
    The \textbf{Header} block also uses a phi node to select the value of the accumulator going into
    \textbf{Body}, either the initial value from \textbf{Entry} or the value calculated in \textbf{Body}
    from the last loop.
    \item \textbf{Body} -
    First we decrement the count and build a thunk for it to allow it to be passed into the user provided
    recursion function.
    We then create a new thunk which calls that function (which is actually a closure) with the count
    and accumulator thunks to get the next value of the accumulator.
    \item \textbf{Exit} -
    Finally we run the accumulator thunk to collapse it into a value which is available for evaluation
    of the rest of the program.
\end{itemize}
