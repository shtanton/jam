\documentclass[12pt,a4paper,titlepage]{article}
\usepackage{mathtools}
\usepackage{stmaryrd}
\usepackage{amssymb}
\usepackage{hyperref}
\usepackage{array}
\usepackage[margin=1in]{geometry}
\usepackage{ebproof}
\usepackage{alltt}

\newcommand{\pipe}{\ |\ }
\newcommand{\typerule}[2]{
    $\dfrac{
        #1
    }{
        #2
    }$
}
\newcommand{\judgement}[1]{
    \Gamma \vdash #1
}

\title{Defining the Language}
\author{Charlie Stanton}

\begin{document}
    \maketitle

    \section{Inspiration}
    The language is based on the calculus defined by Denney \cite{denney98}
    which in turn is based on the simply typed lambda calculus with product types, $\lambda^{\times \rightarrow}$.
    I will only describe additional features that I have added to Denney's calculus.

    Denney defines a signature to be
    \begin{itemize}
        \item A finite set of ground types, $\gamma$.
        \item A finite set of n-ary constants, $k$, with sorts $\tau_1, ..., \tau_n \rightarrow \tau$. I also include a definition for the operational semantics of the function with each of mine.
        \item A finite set of n-ary predicates, $F$, each with a sequence of types $\tau_1, ..., \tau_n$. I include a definition for predicates too.
        \item A potentially infinite set of axioms.
    \end{itemize}

    I have 2 ground types, $bool$ and $u8$ which are booleans and 8 bit unsigned integers respectively.
    I chose $u8$ instead of $nat$ which Denney used as it guaruntees that the program passed to the SMT solver
    will be decidable, since any variable will have finitely many possible values.
    I have constants defined in Table \ref{tab:constants}
    and predicates defined in Table \ref{tab:predicates}.

    \begin{table}
        \begin{tabular}{|l|l|p{0.508\textwidth}|}
            \cline{1-3}
            $true$, $false$ & $bool$ & Literals for the booleans.\\\cline{1-3}
            $\land$, $\lor$, $\Rightarrow$, $\Leftrightarrow$ & $bool, bool \rightarrow bool$ & Standard binary boolean operators.\\\cline{1-3}
            $\lnot$ & $bool \rightarrow bool$ & Unary not.\\\cline{1-3}
            $0$, ..., $255$ & $u8$ & Literals for the 8 bit unsigned integers.\\\cline{1-3}
            $succ$, $pred$ & $u8 \rightarrow u8$ & The successor and predecessor of a $u8$, which will over/underflow.\\\cline{1-3}
            $+$, $-$ & $u8, u8 \rightarrow u8$ & Addition and subtraction for $u8$, which will over/underflow.\\\cline{1-3}
            $u8rec$ & $\tau, (u8 \rightarrow \tau \rightarrow \tau), u8 \rightarrow \tau$ &
            Iterate over the $u8$s less than a given $u8$ in ascending order.\newline
            $u8rec\ t\ t'\ 0 \coloneqq t$ \newline
            $u8rec\ t\ t'\ succ(n) \coloneqq t'\ n\ (u8rec\ t\ t'\ n)$\\\cline{1-3}
        \end{tabular}
        \caption{Constants}
        \label{tab:constants}
    \end{table}

    \begin{table}
        \begin{tabular}{|l|l|p{0.725\textwidth}|}
            \cline{1-3}
            $bool$ & $bool$ & Interpret a boolean expression as a proposition.\\\cline{1-3}
            $\textgreater$, $\textless$, $\leq$, $\geq$ & $u8, u8$ & Integer inequalities.\\\cline{1-3}
        \end{tabular}
        \caption{Predicates}
        \label{tab:predicates}
    \end{table}

    \subsection{Axioms}
    My $u8rec$ is based on Denney's $natrec$, so has 3 axioms schemas. For each $\phi$

    \begin{center}
        \typerule{
            \judgement{n: u8} \quad
            \judgement{t: \phi[0]} \quad
            \judgement{t': \Pi_{n:nat} \Pi_{y:\phi[n]} \phi[succ(n)]}
        }{
            \judgement{u8rec\ t\ t'\ n: \phi[n]}
        }

        \vspace{2mm}
        \typerule{
            \judgement{t: \phi[0]} \quad
            \judgement{t': \Pi_{n:nat} \Pi_{y:\phi[n]} \phi[succ(n)]}
        }{
            \judgement{u8rec\ t\ t'\ 0 =_{\phi[0]} t}
        }

        \vspace{2mm}
        \typerule{
            \judgement{t: \phi[0]} \quad
            \judgement{t': \Pi_{n:nat} \Pi_{y:\phi[n]} \phi[succ(n)]}
        }{
            \judgement{u8rec\ t\ t'\ succ(n) =_{\phi[succ(n)]} t'\ n\ (u8rec\ t\ t'\ n)}
        }
    \end{center}

    \section{Syntax}
    I add several pieces of syntactic sugar which make the language much more usable.

    $\lambda x_1: \phi_1, ..., x_n: \phi_n . t \coloneqq \lambda x_1: \phi_1 .\ ...\ \lambda x_n: \phi_n . t$

    $let\ x_1: \phi_1 =t_1, ..., x_n: \phi_n =t_n\ in\ u \coloneqq (\lambda x_1: \phi_1, ..., x_n: \phi_n . u)\ t_1\ ...\ t_n$

    \section{Operational Semantics}
    Denney only focuses on the typing of his calculus, and doesn't describe operational semantics.
    Since I'm writing a compiler which will produce executables, I need to decide on runtime semantics.
    I have based these on the simply typed lambda calculus with product types.

    \renewcommand{\arraystretch}{2}
    \begin{tabular}{>{$}r<{$}@{ $\rightarrow_\beta$ }>{$}l<{$}}
        (\lambda x:\phi . t)\ t' & t[t'/x]\\
        \pi_i (\langle t_1, t_2 \rangle) & t_i\\
    \end{tabular}
    \renewcommand{\arraystretch}{1}

    Because of the similarity to the simply typed lambda calculus, it is easy to see that any term which can be
    typed must be strongly normalizing, and thus, all functions are total.
    The semantics of propositions are simply those of first order logic, which I won't repeat here,
    with the exception of $\phi \sqsubseteq \psi$ which means that $\phi$ is a supertype of $\psi$.

    \section{Type Checking}
    Denney formally defines the typing rules of his calculus, but an implementation using then is inpractical.
    Should we want to type check $\judgement{\lambda x: \phi . t: \psi}$, the normal route for a syntax directed
    approach is to use the (ABS) rule. However, there are examples where our next step must be (EQ).

    \begin{center}
        \begin{prooftree}
            \hypo{\Gamma, x: \phi \vdash t: \psi}
            \infer1[(ABS)]{\judgement{\lambda x: \phi . t : \Pi_{x:\phi} \psi}}
        \end{prooftree} \quad
        \begin{prooftree}
            \hypo{\judgement{t =_\phi t'}}
            \infer1[(EQ)]{\judgement{t: \phi}}
        \end{prooftree}
    \end{center}

    Reducing the problem using the (EQ) rule is very difficult as we need to produce a suitable $t'$ that permits
    the generation of the proof tree we desire.
    Also, the judgement we are trying to prove is now a proposition (a formula in first-order logic) instead of a
    typing judgement, which requires an algorithm much more complex than the scope of this project to verify
    efficiently.
    To enable the compiler to check the types of a program, we use a translation a program for an SMT solver.
    The first benefit of this approach is that the challenging process of checking a logical formula is handled
    by an SMT solver which I don't need to write, and the second is that SMT solvers can find a model for a formula
    as well as giving a boolean result for if the program is satisfiable.
    We can give the SMT solver a program that causes it to look for possible environments that fit the context
    which fail the judgement, and then show one of those to the programmer to give them an example value for
    a variable which prevents the type checker passing.
    This makes debugging the refinement types much simpler for the programmer.
    I attach some new meaning to the judgement $\judgement{t: \phi}$, which is interpreted as

    \begin{quote}
        For any environment that is in the set of accepted values for the context,
        fully evaluating $t$ using $\beta$ and $\eta$ reduction produces a value which is in the set of accepted
        values for $\phi$.
        Also any term passed to $\pi_i$, a constant, a predicate or a function must evaluate to something that
        is in the set of accepted values for the type guard.
    \end{quote}

    For example $n:(n:u8)n=_{u8} 4 \vdash (\lambda x: (x:u8)\ even(x) . true)\ n : bool$:
    \begin{itemize}
        \item $x$ is accepted by the type $u8$ for $even$.
        \item $n$ is accepted by the type $(x:u8)\ even(x)$.
        \item $\lambda x: (x:u8)\ even(x) . true)\ n$ reduces to $true$ which is accepted by $bool$.
    \end{itemize}
    So we conclude that the judgement is derivable using Denney's rules, which indeed it is.
    I write $\Gamma$ for $n: is4$, $even$ for $(x:u8)\ even(x)$ and $is4$ for $(x:u8)\ x=_{u8}4$.
    I have left out the trivial detail for brevity.

    \begin{center}
        \begin{prooftree}[separation=1em,label separation=0.2em,right label template=(\inserttext)]
            \hypo{...}
            \infer1[CONST]{\Gamma, x: even \vdash true: bool}
            \infer1[ABS]{\judgement{\lambda x: even . true: \Pi_{x:even} bool}}
            \hypo{...}
            \infer1[VAR]{\judgement{n: is4}}
            \hypo{...}
            \infer1{\judgement{even \sqsubseteq is4}}
            \infer2[WEAKEN]{\judgement{n: even}}
            \infer2[APP]{\judgement{(\lambda x: even . true)\ n: bool}}
        \end{prooftree}
    \end{center}

    This approach requires a definition of a set of values which are accepted by a type.
    Fortunately, this is easily defined using the definitions Denney originally assigns to $\Pi_{x:\phi} \psi$
    and $\Sigma_{x:\phi} \psi$ which are of the form $(f: \tau \rightarrow \sigma)P$ and
    $(z: \tau \times \sigma)P$ respectively.
    We can use these to define the set of values accepted by a type by first reducing the type to the form
    $(x: \tau)P$, then the set of accepted values is $\{x \in \tau | P\}$, using the obvious translation
    to treat $\tau$ as a set.

    We can use an SMT solver to see if every possible value of a term $t$ satisfies a condition $P$ (we check
    if it is in the base type $\tau$ separately using standard syntax driven type checking).
    Since finding a possible value of $t$ that does not satisfy $P$ is useful for debugging, we invert $P$ in
    our SMT program and the type checker will pass if the solver yields unsat.
    So we task the SMT solver with finding an environment for which $P[t/x]$ does not hold.

    \subsection{Examples}
    Denney uses the example of a $div2$ function, which makes use of his $natrec$ function.
    I'll use the same example but with $u8rec$ instead. Using
    \[
        div2 \coloneqq \lambda n:u8 . \pi_1 (div2'\ n)
    \]\[
        div2' \coloneqq u8rec\ \langle 0, 0 \rangle\ (\lambda x: u8 . \lambda p: u8 \times u8 .
        \langle \pi_2 (p), \pi_1 (p)+1 \rangle )
    \]
    We want to show
    \[
        \langle \rangle \vdash div2 : \Pi_{n:u8}(m:u8)n =_{u8} 2m \lor n =_{u8} 2m+1
    \]\[
        \langle \rangle \vdash div2' : \Pi_{n:u8} \Sigma_{(m:u8)n =_{u8} 2m \lor n =_{u8} 2m+1} (m': u8) m+m'=n
    \]
    Denney already shows how these can be derived using the typing rules, so we can translate the judgements to SMT
    and expect it to tell us that they are sound.
    My algorithm translates each typing to a new SMT program, which means running it by hand is tedious, I opt to
    demonstrate the most interesting example which is that $div2'$ is in the set described by its type.
    We first seek to find the formula that constrains the set for the type of $div2'$, using Denney's
    expansion, we see that the type of $div2'$ is equivalent to
    \[
        (f: u8 \rightarrow u8 \times u8)P
    \]\[
        P \coloneqq \forall n \in u8 .
        (n = 2 \pi_1 (f n) \lor n = 2 \pi_1 (f n) + 1)
        \land \pi_1 (f n) + \pi_2 (f n) = n
    \]
    We now use the SMT solver to seek a possible environment where this is not true.
    \begin{alltt}
        g x p := <second p, (first p) + 1>
        f n := u8rec <0, 0> g n

        (assert (not (forall ((n u8))
          (and
            (or
              (= n (* 2 (first (f n))))
              (= n (+ (* 2 (first (f n))) 1)))
            (= n (+ (first (f n)) (second (f n))))))))
    \end{alltt}

    I'll go into more detail on the translation to an SMT program later.

    \bibliographystyle{abbrv}
    \bibliography{sources}
\end{document}
