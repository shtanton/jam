\documentclass[12pt,a4paper,titlepage]{article}
\usepackage{mathtools}
\usepackage{stmaryrd}
\usepackage{amssymb}
\usepackage{hyperref}
\usepackage{array}
\usepackage[margin=1in]{geometry}

\newcommand{\pipe}{\ |\ }

\title{Defining the Language}
\author{Charlie Stanton}

\begin{document}
    \maketitle

    \section{Inspiration}
    The language is based on the calculus defined and explored in this paper:

    \url{https://www.researchgate.net/profile/Ewen-Denney/publication/2406398_Refining_Refinement_Types/links/56d3210e08ae059e376122e4/Refining-Refinement-Types.pdf}

    Which in turn is based on the simply typed lambda calculus with product types, $\lambda^{\times \rightarrow}$.

    \section{Syntax}
    The language has 3 fundamental constructs

    \begin{itemize}
        \item Expressions which evaluate to values at runtime
        \item Types which restrict the possible values that an expression can have
        \item Propositions which can be used in types for much finer control over the type restrictions than most languages offer
    \end{itemize}

    These are defined using the following grammar

    \renewcommand{\arraystretch}{2}
    \begin{tabular}{l>{$}c<{$}>{$}c<{$}}
        Expressions & t \coloneqq&
        x \pipe (k\ t_1\ ...\ t_n) \pipe \ast \pipe <t_1\ t_2> \pipe (fn\ x:\phi\ t) \pipe (\pi_i\ t) \pipe (t\ t')\\
        Types & \phi \coloneqq&
        \textbf{1} \pipe \gamma \pipe <x:\phi\ \psi> \pipe (fn\ x:\phi\ \psi) \pipe x:\phi\{P\}\\
        Propositions & P \coloneqq&
        false \pipe (=>\ P\ Q) \pipe (forall\ x:\phi\ P) \pipe (F\ t_1\ ...\ t_n)\\
        &&\pipe (=\ \phi\ t_1\ t_2) \pipe (:>\ \phi\ \psi)
    \end{tabular}
    \renewcommand{\arraystretch}{1}

    \section{Operational Semantics}
    The operational semantics of the language match that of the simply typed lambda calculus with product types.
    Note that the language is call-by-name, which greatly simplifies the type checker.

    \renewcommand{\arraystretch}{2}
    \begin{tabular}{>{$}r<{$}@{ $\rightarrow_\beta$ }>{$}l<{$}}
        ((fn\ x:\phi\ t)\ t') & t[t'/x]\\
        (\pi_i\ <t_1\ t_2>) & t_i\\
    \end{tabular}
    \renewcommand{\arraystretch}{1}

    Because of the similarity to the simply typed lambda calculus, it is easy to see that any term which can be
    typed must be strongly normalizing, and thus, all functions are total.
    The semantics of propositions are simply those of first order logic, which I won't repeat here,
    with the exception of $(:> \phi \psi)$ which means that $\phi$ is a supertype of $\psi$.

    \section{Typing rules}
    The typing rules of the language are formally defined in the appendix of the paper references earlier.
    As well as the formal rules, we can also assign a more practical meaning to the typing judgements.
    For these to make sense we need a notion of an environment which can be accepted by a context $\Gamma$.
    An environment is a value for every variable in the context, and it is accepted by the context of each
    value is accepted by the type given to it by the context.

    \renewcommand{\arraystretch}{2}
    \begin{tabular}{>{$}r<{$}p{0.7\textwidth}}
        \Gamma \vdash t : \phi &
        For any environment that is accepted by $\Gamma$, evaluating $t$ will produce a value which is acceptable
        to $\phi$. Also any arguments to applications of functions, constants, $\pi_i$ or predicates have values
        which are acceptable to the type guards on those applications.
        \\
        \Gamma \vdash P &
        $P$ is true for any environment that is accepted by $\Gamma$.
        \\
    \end{tabular}
    \renewcommand{\arraystretch}{1}

    The types $<x:\phi\ \psi>$ and $(fn\ x:\phi\ \psi)$ can both be reduced to simple types with refinements
    that will accept the same set of values.
    In fact, any type can be written as $x:\sigma\{P\}$ where $\sigma$ is a simple type.
    A type $x:\sigma\{P\}$ accepts the value of $t$ if both:
    \begin{itemize}
        \item In the simply typed lambda calculus, we have that $\sigma$ accepts $t$.
        \item The proposition $P[t/x]$ is true.
    \end{itemize}
    The extra condition that the proposition must be true is the only thing that separates the refinement typed
    language from a simply typed language, but it is extremely powerful.
\end{document}
