\documentclass[12pt,a4paper,titlepage]{article}
\usepackage{mathtools}
\usepackage{stmaryrd}
\usepackage{amssymb}

\newcommand{\lnc}[1]{Logic(\langle \rangle \vdash #1)}

\title{Type checking the refinement typed calculus with an SMT solver}
\author{Charlie Stanton}

\begin{document}
    \maketitle

    \section{Starting Calculus}

    \section{Algorithm Overview}
    To simplify the algorithm, I have split it into several sections:

    \begin{itemize}
        \item Check some of the unrefined types
        \begin{itemize}
            \item Check arguments to constants
            \item Check arguments to predicates
            \item Check arguments to $\pi_i$
        \end{itemize}
        \item Generate all required type judgements
        \begin{itemize}
            \item Find all applications in the program
            \item Find the type expected for the argument passed to each function that's applied
        \end{itemize}
        \item Translate each type judgement to logical formulae
        \item Translate each logical formula into a simple formula
        \begin{itemize}
            \item Remove refinement types
            \item Lambda lift the program
            \item Remove pairs from program
            \item Specialize functions to remove function parameters
            \item Uncurry functions to remove function returns
        \end{itemize}
        \item Translate the formulae into SMT programs
    \end{itemize}

    Higher order functions need to be removed as the SMT solver needs to have an equivalent expression
    for every expression in the program, however the SMT solver has no support for higher order functions
    or anonymous lambdas.

    I will carry an example through my explanation of each of the steps, which will be:

    $((\lambda f: \Pi_{x: even} odd . (\lambda x: even . f x)) (\lambda x: even . S x)) 0$

    \section{Some syntactic sugar}
    To make things simpler, I define several shorthands. Note that these are for use in the refinement calculus, since the logic already has them.

    $true \coloneqq \ast =_\textbf{1} \ast$

    $\neg P \coloneqq P \supset false$

    $P \lor Q \coloneqq \neg P \supset Q$

    $P \land Q \coloneqq \neg (P \supset \neg Q)$

    \section{Finding Applications}

    This is the first step in the actual type checking, since some steps are just to make the encoding to SMT
    possible.
    The objective is to find every application in the program as these are where the type checks need to happen.
    Continuing the example, we have:

    $f: \Pi_{x: even} odd, x: even \vdash f x$

    $\langle \rangle \vdash (\lambda f: \Pi_{x: even} odd . (\lambda x: even . f x)) (\lambda x: even . S x)$

    $\langle \rangle \vdash ((\lambda f: \Pi_{x: even} odd . (\lambda x: even . f x)) (\lambda x: even . S x)) 0$

    \section{Find the expected types of arguments to functions applied}

    Now we have every location, we need to know the expected type of the argument passed so we can check that
    the argument does in fact have that type.
    This uses a recursive function taking 2 arguments, the expression to find the parameter type for,
    and a stack of expressions that are applied to the expression first before the application we're investigating.
    This function is possible as refinements on types can't affect the expected type of the argument.

    TODO: PROOF OF THAT CLAIM

    It is defined as follows:

    $ArgType(\Gamma, f: \Pi_{x: \phi} \psi, \Gamma' \vdash f, nil) := \phi$

    $ArgType(\Gamma, f: (f: \phi)P, \Gamma' \vdash f, R) := ArgType(\Gamma, f: \phi, \Gamma' \vdash f, R)$

    $ArgType(\Gamma \vdash \lambda x: \phi . t, nil) := \phi$

    $ArgType(\Gamma, f: \Pi_{x: \phi} \psi, \Gamma' \vdash f, t : R) := ArgType(\Gamma, f': \psi[t / x], \Gamma' \vdash f', R)$

    $ArgType(\Gamma \vdash \lambda x: \phi . t, t' : R) := ArgType(\Gamma \vdash t[t'/x], R)$

    $ArgType(\Gamma \vdash t t', R) := ArgType(\Gamma \vdash t, t' : R)$

    For some application $\Gamma \vdash t t'$, we are interested in verifying the typing judgement $\Gamma \vdash t': ArgType(t, nil)$.

    Continuing the example, we have:

    $f: \Pi_{x: even} odd, x: even \vdash x: even$

    $\langle \rangle \vdash \lambda x: even . S x: \Pi_{x: even} odd$

    $\langle \rangle \vdash 0: even$

    \section{Translation to Logic}

    The translation to logic makes use of the $Logic$ function, which is defined as follows.

    $Logic(x: \phi, \Gamma \vdash t: \psi) \coloneqq
    \forall x \in Simplify(\phi) . \lnc{x: \phi} \Rightarrow Logic(\Gamma \vdash t: \psi)$

    $\lnc{t: \textbf{1}} \coloneqq is\textbf{1}\ Logic(t)$

    $\lnc{t: \gamma} \coloneqq is\gamma\ Logic(t)$

    $\lnc{t: \Sigma_{x: \phi} \psi} \coloneqq \lnc{\pi_1 (t) : \phi} \land \lnc{\pi_2 (t): \psi[\pi_1 (t) / x]}$

    $\lnc{t: \Pi_{x: \phi} \psi} \coloneqq \forall x \in Simplify(\phi) . \lnc{x: \phi} \Rightarrow \lnc{tx: \psi}$

    $\lnc{t: (x: \phi)P} \coloneqq \lnc{t: \phi} \land \lnc{P[t/x]}$

    $\lnc{false} \coloneqq false$

    $\lnc{P \supset Q} \coloneqq \lnc{P} \Rightarrow \lnc{Q}$

    $\lnc{\forall x: \phi . P} \coloneqq \forall x \in Simplify(\phi) . \lnc{x: \phi} \Rightarrow \lnc{P}$

    $\lnc{F(t_1, ..., t_n)} \coloneqq F(Logic(t_1), ..., Logic(t_n))$

    $\lnc{t =_\textbf{1} t'} \coloneqq \lnc{t: \textbf{1}} \land \lnc{t': \textbf{1}}$

    $\lnc{t =_\gamma t'} \coloneqq \lnc{t: \gamma} \land Logic(t) = Logic(t')$

    $\lnc{t =_{\Sigma_{x: \phi} \psi} t'} \coloneqq
    \lnc{\pi_1 (t) =_\phi \pi_2 (t')} \land \lnc{\pi_2 (t) =_{\psi[t/x]} \pi_2 (t')}$

    $\lnc{t =_{\Pi_{x: \phi} \psi} t'} \coloneqq
    \forall x \in Simplify(\phi) . \lnc{x: \phi} \Rightarrow \lnc{tx =_\psi t'x}$

    $\lnc{t =_{(x: \phi)P} t'} \coloneqq \lnc{t =_\phi t'} \land \lnc{P[t/x]} \land \lnc{P[t'/x]}$

    $Logic(x) \coloneqq x$

    $Logic(k(t_1, ..., t_n)) \coloneqq k(Logic(t_1), ..., Logic(t_n))$

    $Logic(\ast) \coloneqq \ast$

    $Logic(\langle t, t' \rangle) \coloneqq \langle Logic(t), Logic(t') \rangle$

    $Logic(\lambda x: \phi . t) \coloneqq \lambda x: Simplify(\phi) . Logic(t)$

    $Logic(\pi_i (t)) \coloneqq \pi_i (Logic(t))$

    $Logic(t\ t') \coloneqq Logic(t)\ Logic(t')$

    These operations also rely on the function $Simplify$ defined as follows.

    $Simplify(\textbf{1}) \coloneqq Term$

    $Simplify(\gamma) \coloneqq Term$

    $Simplify(\Sigma_{x: \phi} \psi) \coloneqq \langle Simplify(\phi), Simplify(\psi) \rangle$

    $Simplify(\Pi_{x: \phi} \psi) \coloneqq Simplify(\phi) \rightarrow Simplify(\psi)$

    $Simplify((x: \phi)P) \coloneqq Simplify(\phi)$

    \section{Lambda Lifting}
    This is the simplest step, so I go into the least detail.
    The objective is to remove lambdas from the program and use only globally defined functions, as these are all
    that is available in the SMT solver.
    Whenever there is an abstraction expression, we define a new global function which takes the parameter of the
    original abstraction, but also a parameter for each environment variable it requires, which can be deduced at
    compile time.
    Then the abstraction is replaced with the global function partially applied with the environment variables it
    requires. For example:

    $((\lambda f: nat \rightarrow nat . (\lambda x: nat . f x)) (\lambda x: nat . S x)) 0$

    becomes:

    $fn_1\ (f: nat \rightarrow nat)\ (x: nat) := f\ x$

    $fn_2\ (f: nat \rightarrow nat) := fn_1\ f$

    $fn_3\ (x: nat) := S x$

    $main := fn_2\ fn_3\ 0$

    TODO: FORMAL DEFINITION OF LAMBDA LIFTING OR REFERENCE TO PAPER CONTAINING ONE GOES HERE

    \section{Remove Pairs}

    The SMT solver has algebraic datatypes which can be used as records such as pairs, however we need to remove
    all higher order functions using specialization and uncurrying, and those processes won't work if there
    are functions inside pairs.
    To simplify the type checker, we remove all pairs, so there are no pair types, or $\langle t, t' \rangle$ or
    $\pi_i (t)$ expressions.

    Use the same data structure as the specialization. Iterate through the functions starting with the first.
    For this stage we temporarily introduce some new semantics, which simplify the transformation:

    $\langle t_1, t_2 \rangle t' \rightarrow_\beta \langle t_1\ t', t_2\ t' \rangle$

    Introducing this doesn't change the original program as if there were any instances of this structure before
    then the type checker would've failed when finding the types of arguments.
    The process is split into steps which, thanks to the new semantics, all produce equivalent programs,
    however before and after the process as a whole, the semantics aren't necessary.
    First we aim to remove pairs returned from functions. We do this by checking the return type of each function
    and if it is a pair we replace it with 2 new functions, going from

    $f\ x_1\ ...\ x_n := t$

    to

    $f_1\ x_1\ ...\ x_n := \pi_1(t)$

    $f_2\ x_1\ ...\ x_n := \pi_2(t)$

    Also any parameters with the type of
    $\sigma_1 \rightarrow ... \rightarrow \sigma_n \rightarrow \langle \tau_1, \tau_2 \rangle$
    should have their types changed to
    $\langle \sigma_1 \rightarrow ... \rightarrow \sigma_n \rightarrow \tau_1, \sigma_1 \rightarrow ... \rightarrow \sigma_n \rightarrow \tau_2 \rangle$
    We then replace any instances of $f$ in the whole program with $\langle f_1, f_2 \rangle$.
    Because of the new semantics, this new program is equivalent to the old one.

    Next we aim to remove pairs as parameters to functions. We do this by checking the types of any parameters
    for functions, and the types of parameters of highers order parameters. We change

    $f\ x_1\ ...\ (y: \langle \sigma_1, \sigma_2 \rangle)\ ...\ x_n := t$

    to

    $f\ x_1\ ...\ (y_1: \sigma_1)\ (y_2: \sigma_2)\ ...\ x_n := t[\langle y_1, y_2 \rangle / y]$

    and changing parameters types of higher order parameters from

    $\sigma_1 \rightarrow ... \rightarrow \langle \tau_1, \tau_2 \rangle \rightarrow ... \rightarrow \sigma_n \rightarrow \sigma$

    to

    $\sigma_1 \rightarrow ... \rightarrow \tau_1 \rightarrow \tau_2 \rightarrow ... \rightarrow \sigma_n \rightarrow \sigma$

    Now there are no functions that accept pairs as arguments, so we must change everywhere a pair is passed as
    an argument.
    For any instance of $t\ t'$ where $t'$ is of type $\langle \tau_1, \tau_2 \rangle$ we replace it with
    $(t\ \pi_1(t'))\ \pi_2(t')$.
    Now no pairs are being passed as arguments, and the program is still equivalent.

    Next we remove uses of the new semantics by expanding them using the new rule.
    Any occurances of $\langle t_1, t_2 \rangle t'$ are replaced with $\langle t_1\ t', t_2\ t' \rangle$.

    There are now very few uses of pairs left.
    All abstractions were already removed and $\ast$ and constants couldn't be pairs anyway.
    Additionally:
    \begin{itemize}
        \item No parameters are pairs so a pair cannot be a variable
        \item No function returns a pair so a pair cannot be an application
    \end{itemize}

    Also places where the pair syntax ($\langle t_1, t_2 \rangle$) is used are now very limited too, as
    it cannot be on either side of an application anymore.
    If we consider the innermost usage of $pi_i(t)$ in a function, then the $t$ must have the structure
    $\langle t_1, t_2 \rangle$ meaning we can simplify $pi_i(\langle t_1, t_2 \rangle)$ to $t_i$.
    This removes all occurances of $\pi_i$ so the only remaining possible structure of a pair is
    $\langle t_1, t_2 \rangle$. We can prove that none of these exist by contradiction.

    Suppose one of these exists in a function, there must be an outermost usage that isn't contained in another
    usage, but it must be contained in something since the function can't return it. It can't be in a constant,
    application or $\pi_i$ so must be inside another pair literal, which contradicts our assumption.

    Thus we have removed all occurances of pair types, pair literals and $\pi_i$ from the program while keeping
    it equivalent to the original program.

    \section{Specialize Functions}
    Since the SMT solver can't reason with functions that are passed as parameters, we must find an equivalent
    program that doesn't have higher order parameters (HOPs).
    There are several ways of doing this, the most widely applicable is defunctionalization, which would work,
    but specialization is much simpler.
    The disadvantage of specialization is that in some recursive cases it goes into an infinite loop, however,
    the calculus for Jam doesn't support recursion, so this isn't a problem.

    Anywhere a function is passed as an argument, define a new function based on the one it is being passed
    to but "specialized" to that argument, and change the application to use the new function.

    Continuing our example, changing the application in $main$ produces:

    $main := fn_2 fn_3\ 0$

    $fn_2 fn_3 := fn1\ fn3$

    Then changing the application in $fn2fn3$ produces:

    $fn_2 fn_3 := fn_1 fn_3$

    $fn_1 fn_3\ (x: nat) := fn_3\ x$

    So removing functions that are no longer used, the new program is:

    $fn_1 fn_3\ (x: nat) := fn_3\ x$

    $fn_2 fn_3 := fn_1 fn_3$

    $fn_3\ (x: nat) := S x$

    $main := fn_2 fn_3\ 0$

    Algorithm:

    The program is stored as a list of function definitions.
    Each function has a list of arguments with basic types and a body expression.
    Each function body only references functions defined strictly before it in the list (this is
    possible as there is no recursion or mutual recursion in Jam)

    Starting at the end of the list of functions and working backwards, traverse the function body depth first
    looking for an application where a function is used as an argument.
    Inductively, since the process has already been run on all functions that reference this function, this
    function doesn't take functional parameters.
    Since constants can only return primitive types and can't be partially applied or passed as arguments,
    the argument isn't a constant or constant call. Obviously $\ast$ isn't a function, and $\langle t_1, t_2 \rangle$,
    $\pi_i (t)$ and $\lambda x: \phi . t$ expressions have all been removed by previous transformations.
    Thus we know that the shape of the argument will be $f t_1 ... t_n$ where $f$ is a function defined earlier
    in the list and $t_1, ..., t_n$ are expressions (where it is possible for $n$ to equal 0).

    So with this in mind, we have some case of

    $f\ x_1\ ...\ x_m := C[g\ u_1\ ...\ u_p\ (h\ t_1\ ...\ t_n)]$

    For some naturals $m$, $n$, $p$ and some context $C$ and the function $g$ earlier in the list

    $g\ y_1\ ...\ y_p\ y := t$

    Define immediately before function $f$ in the list a new function $g'$

    $g'\ y_1\ ...\ y_p\ z_1\ ...\ z_n := t[h\ z_1\ ...\ z_n / y]$

    and change $f$ to

    $f\ x_1\ ...\ x_m := C[g'\ u_1\ ...\ u_p\ t_1\ ...\ t_n]$

    This removes one HOP from the program, so simply repeat this until there are no remaining HOPs.

    \section{Uncurrying}

    Using the same data structure as for the specialization, but now all HOPS are gone.
    Some functions still might return functions, in which case we simply use $\eta$ equality to add
    another parameter and apply the body to said parameter to get the new body.
    After this all of the parameters for each function will explicit and the bodies will all have
    primitive types. This operation might need to be done sometimes while specializing so functions have enough
    parameters to add the new parameters.

    We have now removed functions as arguments for or being returned from functions, which means there are no
    possible remaining partial applications of functions, as the only place a function can be referenced is where
    it is applied.
    This means we can uncurry all the functions, removing the ability to partially apply them.

    Continuing the example:

    $fn_1 fn_3\ (x: nat) := fn_3\ 0$

    $fn_2 fn_3\ (x: nat) := fn_1 fn_3\ x$

    $fn_3\ (x: nat) := S x$

    $main := fn_2 fn_3\ 0$

    \section{Construct the SMT Program}

    At this point we have a set of logic formulae which can be understood by the SMT solver, and we want to be
    true for the program to pass type checking.
    In this last stage we first invert the formulae since we want the SMT solver to return UNSAT on success.
    This means we can pull lots of existential quantifiers out to the front of each formula, which represent
    the environment of the typing judgement.
    This means that if the SMT solver returns SAT then it will also give us a model environment that could exist
    and which would violate the refinements, which is incredibly useful for debugging.

\end{document}
